\mtcaddchapter[Installation]                          % solution pour minitoc
\markboth{\uppercase{Installation}}{\uppercase{Installation}} 

\section*{Requierement}

\mudiff requires Matlab software, version 2011 or higher. \mudiff should work fine with previous version of Matlab, however without guarantee. Please note that Matlab is not provided with \mudiff, it can be bough and downloaded at \url{http://www.mathworks.com/}.

\section*{Install}

The install process is done as follows:
\begin{enumerate}
\item Download the \mudiff toolbox, either using \texttt{git} with
\begin{verbatim}
git clone http://mu-diff.math.cnrs.fr/git mu-diff
\end{verbatim} 
or by downloading the following zip file and unzipping it in your matlab working directory (or whereelse you prefer):
\begin{center}
\url{http://mu-diff.math.cnrs.fr/mu-diff/Download_files/mudiff.zip}
\end{center}
Note that \texttt{git} should be prefered to stay up to date easily.
\item Add the \mudiff toolbox to Matlab's path (including subfolders !), using either the graphical interface of Matlab or the \code{addpath} and \code{save path} functions.
\item Test the \mudiff install by typing in the Matlab command window:
\begin{lstlisting}
BenchmarkDirichlet;
\end{lstlisting}
This should solve the multiple scattering problem using classical boundary integral equation, described in chapter \ref{chap:math}, with in addition, the radar cross section and the history of GMRES of the boundary integral equations.
\item Other test can be launched to be sure every integral operators are working fine:
\begin{lstlisting}
BenchmarkNeumann;
BenchmarkPenetrable;
BenchmarkCalderon;
\end{lstlisting}
\item If everything went right: congratulation! Your \mudiff installation is working!
\end{enumerate}

If you want to see some other examples using \mudiff, you can try and launch the following time reversal experiments:
\begin{lstlisting}
DORT_NotPenetrable;
DORT_dielectric;
\end{lstlisting}
