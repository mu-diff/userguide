\section{Spectral formulation used in $\mu$-diff}\label{MuDiffFormulation}
\label{secEqInt:disque}

We consider now  circular cylinders as scatterers. In this situation, we can explicitly compute  the boundary integral
equations in a Fourier basis,  leading therefore to an efficient computational spectral method when used in conjunction
with numerical linear algebra methods (direct or iterative solvers).
%====================================
        \subsection{Notations and Fourier basis}\label{secEqInt:BaseFourier}
%====================================

Let us consider an orthonormal system $(\OO,\V{\OO x_{1}},\V{\OO x_{2}})$. We assume that the scattering obstacle
 $\Omegam$ is the union of  $M$ disks $\Omegamp$, for $p = 1,\ldots,M$, of radius $a_p$ and center $\OOp$.
 We define $\Gamma_p$ as the boundary of  $\Omegamp$ and by $\dsp{\Gamma = \cup_{p=1 \ldots M}\Gamma_p}$ the boundary of
  $\Omegam$.  The unit normal vector $\nn$ to $\Omegam$ is outgoing. An illustration of the notations is reported on
  Figure \ref{figEqInt:schemanotations}.
  
  \begin{figure}[h!]
\begin{center}
\def\svgwidth{10cm}
\import{./img/ChapEqInt/disques/}{schema.pdf_tex}
\end{center}
\caption{Illustration of the notations for two disks $\Omegamp$ and $\Omegamq$ and a point $\xx�\in \Omegaps$.}
\label{figEqInt:schemanotations}
\end{figure}
  
For any $p=1, \ldots, M$, we introduce $\bbp $ as the vector between the center $\OOp$ and the origin $\OO$
$$%\begin{equation}\label{eqEqInt:bp}
\bbp = \OO \OOp, \qquad \qquad b_p= \left\|\bbp\right\|, \qquad\qquad \alpha_p=Angle(\V{\OO x_{1}}, \bbp),
$$%\end{equation}
and, for $q = 1,\ldots,M$, with $q \neq p$, $\bbpq$ as the vector between the centers $\OOq$ and $\OOp$ 
$$%\begin{equation}\label{eqEqInt:bpq}
\bbpq = \OOq\OOp, \qquad \qquad b_{pq}=\left\|\bbpq  \right\|, \qquad\qquad
\alpha_{pq}=Angle(\V{\OO x_{1}},\bbpq ).
$$%\end{equation}
Furthermore, any point $\xx$ is described by its global polar coordinates
$$
\rr (\xx)= \OO \xx,  \qquad \qquad r (\xx)=\left\|\rr(\xx) \right\|,  \qquad\qquad
\theta(\xx)=Angle(\V{\OO x_{1}},\br(\xx)),
$$
or by its polar coordinates in the orthonormal system associated with the obstacle $\Omegamp$, with $p =1, \ldots, M$, 
$$%\begin{equation}\label{eqEqInt:PolaireLocal}
\rrp (\xx)= \OOp \xx, \qquad \qquad r_p(\xx)  = \left\|\rrp(\xx)\right\|,  \qquad
\qquad \theta_p(\xx) = Angle(\V{\OOp x_{1}}, \rrp(\xx)).
$$%\end{equation}


Let us now build a basis of  $L^2(\Gamma)$ to approximate the integral operators. To this end, we first construct
 a basis of $L^2(\Gamma_p)$ associated with $\Omegamp$, for
 $p=1,\ldots,M$. If the circle $\Gamma_p$ has a radius one and is centered
 at the origin, then a suitable basis of $L^{2}(\Gammap)$ is the spectral Fourier basis of functions $(e^{im\theta})_{m \in \Zb}$.
 We adapt this basis to the general case where $\ap \neq 1$ by introducing, on one hand, the functions $(\phim)_{m\in\Zb}$ defined on $\Rb^{2}$ by: 
$\forall m\in\Zb$, $\forall \xx \in \Rbb$, $\varphi_m(\xx) = e^{im\theta(\xx)}$,
and, on the other hand,  the functions  $(\phimp)_{1\leqslant p \leqslant M, \; m\in\Zb}$  given by
$$%\begin{equation}\label{eqEqInt:phimp}
\forall p=1,\ldots,M, \forall m\in\Zb, 
\forall \xx \in \Gamma_p, \qquad\varphi_m^p(\xx) = \frac{\varphi_m(\rr_p(\xx))}{\sqrt{2 \pi a_p}} = \frac{e^{im\theta_p(\xx)}}{\sqrt{2 \pi a_p}}.
$$%\end{equation}
For $p=1,\ldots,M$, the family $\dsp{\left( \varphi_m^p \right)_{m \in \Zb}}$ forms an orthonormal basis of
 $L^2(\Gamma_p)$ for the hermitian inner product $\PSGammap{\cdot}{\cdot}$
$$
\forall f,g\in L^{2}(\Gammap),\qquad \PSGammap{f}{g} = \int_{\Gammap} f(\xx) \overline{g(\xx)} \dd\Gammap(\xx).
$$
To build a basis of $L^2(\Gamma)$, we introduce the functions $\Phi_m^p$ of $L^2(\Gamma)$ as the union of these $M$ families 
$$%\begin{equation}\label{eqEqInt:Phimp}
\forall p,q = 1,\ldots, M, \forall m\in\Zb, \qquad \Phi_m^p |_{\Gamma_q} = 
\begin{cases}
        0 & \text{if } q \neq p, \\
        \varphi_m^p & \text{if } q = p.
\end{cases}
$$%\end{equation}
The family $\BF = \{ \Phi_m^p, \, m \in \Zb, p =1, \ldots, M\}$, also called Fourier or spectral basis, is a Hilbert basis of
 $L^2(\Gamma)$ for the usual scalar product $(\cdot,\cdot)_{L^{2}(\Gamma)}$. 


\subsection{Integral operators - integral equations for a cluster of circular cylinders}\label{integralinfinite}


In view of a numerical procedure, $\mu$-diff uses the weak formulation of the EFIE (\ref{EFIED}) in 
 $L^{2}(\Gamma)$ based on the Fourier basis $\BF$ 
$$
\begin{cases}
\text{Find $\rho \in H^{-1/2}(\Gamma)$ such that for any $p=1,\ldots,M,$ and $m\in\Zb$,} & \\
\PSGamma{L\rho}{\Phimp} = -\PSGamma{\uincg}{\Phimp}. &
\end{cases}
$$
Since $\uinc$ is assumed to be smooth enough (typically $\Cscr^{\infty}$) and that $\Gamma$ is $\Cscr^{\infty}$, then
the scattered wavefield is also $\Cscr^{\infty}(\Omegaps)$ and the density  $\rho$ is (at least) in
 $H^{1/2}(\Gamma)$. Therefore, $\rho$ can be expanded in  $\BF$ as 
$$
\rho = \sum_{q=1}^{M}\sum_{n\in\Zb}\rhonq\Phinq$$ and the weak form of the EFIE  is 
$$
\begin{cases}
\text{Find the Fourier coefficients $\rhonq\in\Cb$, for $q=1,\ldots,M$, and $n\in\Zb$, such that,} & \\
\dsp{\forall p=1,\ldots,M, \;\forall m\in\Zb, \qquad \sum_{q=1}^{M}\sum_{n\in\Zb}\rhonq\PSGamma{L\Phinq}{\Phimp} = -\PSGamma{\uincg}{\Phimp}}. &
\end{cases}
$$
This formulation can be written under the following matrix form $
\Lbt\Rhot = \UUt$,
where the infinite matrix representation
 $\Lbt =(\Lbtpq)_{1\leqslant p,q\leqslant M}$ and the infinite vectors
  $\Rhot=(\Rhot^{p})_{1\leqslant p\leqslant M}$ and
   $\UUt=(\UUt^{p})_{1\leqslant p\leqslant M}$ are defined by blocks   as
\begin{equation}\label{eqEqInt:MatLbt}
\Lbt =
\left[
\begin{array}{c c c c}
\Lbt^{1,1} & \Lbt^{1,2} & \ldots & \Lbt^{1,M} \\
\Lbt^{2,1} & \Lbt^{2,2} & \ldots & \Lbt^{2,M} \\
\vdots & \vdots & \ddots & \vdots\\
\Lbt^{M,1} & \Lbt^{M,2} & \ldots & \Lbt^{M,M}
\end{array}
\right],
\qquad \qquad
\Rhot =
\left[
\begin{array}{c}
\Rhot^{1} \\
\Rhot^{2} \\
\vdots \\
\Rhot^{M}
\end{array}
\right],\qquad 
\UUt =
\left[
\begin{array}{c}
\UUt^{1} \\
\UUt^{2} \\
\vdots \\
\UUt^{M}
\end{array}
\right],
\end{equation}
with, for any $p,q=1,\ldots,M$, and $m,n\in\Zb$:
$
\Lbtpqmn = \PSGamma{L\Phinq}{\Phimp}$, $\Rhot_{m}^{p} = \rhomp$ and
$\UUt_{m}^{p} = \PSGamma{-\uincg}{\Phimp}$.

For the other integral formulations (section \ref{AutresEI}) or even for any other boundary condition, 
 the expressions of the three boundary integral operators  $M$, $N$ and $D$ are needed.
Therefore, to compute an integral equation, we introduce the infinite matrices $\Mbt = (\Mbtpq)_{1\leqslant p,q\leqslant M}$, $\Nbt= (\Nbtpq)_{1\leqslant p,q\leqslant M}$ and
 $\Dbt= (\Dbtpq)_{1\leqslant p,q\leqslant M}$, with the same block structure as $\Lbt$ (see equation
  (\ref{eqEqInt:MatLbt})). For $p,q =1,\ldots, M$, the coefficients of the infinite matrices $\Mbtpq$, $\Nbtpq$ and $\Dbtpq$ are defined for any indices
   $m$ and $n$ in $\Zb$ by
\[
\Mbtpqmn = \PSGamma{M\Phinq}{\Phimp},
\Nbtpqmn = \PSGamma{N\Phinq}{\Phimp}, \textrm{ and }
\Dbtpqmn = \PSGamma{D\Phinq}{\Phimp}.
\]
For a numerical implementation, we can explicitly compute \cite{JACT,ThierryThesis} the matrix blocks
 $\Lbtpq$, $\Mbtpq$, $\Nbtpq$ and $\Dbtpq$ involved in $\Lbt$, $\Mbt$, $\Nbt$ and $\Dbt$, for
  $p,q=1,\ldots,M$. To this end, we introduce the infinite diagonal matrices
   $\Jbtp$, $\dJbtp$, $\Hbtp$ and $\dHbtp$, with general terms, for $m \in \Zb$,
$$%\begin{equation}\label{eqEqInt:Jbtp}
\Jbtp_{mm} = \Jm(ka_{p}), \hspace{0.5cm}
\dJbtp_{mm} = \Jmp(ka_{p}), \hspace{0.5cm}
\Hbtp_{mm} = \Hm(ka_{p}), \hspace{0.5cm}
\dHbtp_{mm} = \Hmp(ka_{p}).
$$%\end{equation}
In addition, let $\Ibtp$ be the infinite identity matrix, and, for $q\neq p$, the infinite separation matrix $\Sbtpq$ between the obstacles $\Omegamp$ and
 $\Omegamq$, defined by
$$%\begin{equation}\label{eqEqInt:Sbtpq}
\Sbtpq=(\Sbtpqmn)_{m\in \Zb,n\in
\Zb}\qquad \text{and}\qquad \Sbtpqmn=\Smnbpq = H_{m-n}^{(1)} (k b_{pq})e^{i(m-n)\alpha_{bq}}.
$$%\end{equation}
Under these notations, we rewrite the blocks $\Lbtpq$, $\Mbtpq$, $\Nbtpq$ and $\Dbtpq$ of the infinite matrices
 $\Lbt$, $\Mbt$, $\Nbt$ and $\Dbt$ under the matrix form, for any $p,q= 1,\ldots,M$,
 \begin{itemize}
\item[]\begin{flalign}\label{eq:InfL}
\text{\textbullet}\quad&
\Lbtpq = 
\begin{cases}
\dsp{\frac{i\pi a_p}{2} \Jbtp\Hbtp,} & \text{ if } p = q,\\[0.3cm]
\dsp{\frac{i \pi\sqrt{a_p a_q}}{2} \Jbtp (\Sbtpq)^T\Jbtq,} &\text{ if } p \neq q,
\end{cases}&\end{flalign}\item[]\begin{flalign}\label{eq:InfM}
\text{\textbullet}\quad&
\Mbtpq = 
\begin{cases}
\dsp{ - \frac{1}{2}\Ibtp - \frac{i \pi k a_p}{2} \Jbtp \dHbtp = \frac{1}{2}\Ibtp - \frac{i \pi k a_p}{2} \dJbtp \Hbtp,} &\text{ if } p = q,\\[0.3cm]
\dsp{- \frac{i k \pi\sqrt{a_p a_q}}{2} \Jbtp (\Sbtpq)^T \dJbtq,}&\text{ if } p \neq q,
\end{cases}&\end{flalign}\item[]\begin{flalign}\label{eq:InfN}
\text{\textbullet}\quad&
\Nbtpq =
\begin{cases}
\dsp{\frac{1}{2}\Ibtp + \frac{i \pi k a_p}{2} \Jbtp \dHbtp  = -\frac{1}{2}\Ibtp + \frac{i \pi k a_p}{2} \dJbtp \Hbtp,} &\text{ if } p = q, \\[0.3cm]
\dsp{\frac{i k \pi\sqrt{a_p a_q}}{2} \dJbtp (\Sbtpq)^T \Jbtq,}& \text{ if } p \neq q, 
\end{cases}&\end{flalign}\item[]\begin{flalign}\label{eq:InfD}
\text{\textbullet}\quad&
\Dbtpq =
\begin{cases}
\dsp{\frac{i \pi k^2 a_p}{2} \dJbtp\dHbtp,} & \text{ if } p = q, \\[0.3cm]
\dsp{ - \frac{i k^2 \pi\sqrt{a_p a_q}}{2} \dJbtp (\Sbtpq)^T \dJbtq,} &\text{ if } p \neq q,
\end{cases}&\end{flalign}\end{itemize} 
where $(\Sbtpq)^T$ is the transpose matrix of the separation matrix $\Sbtpq$. 

The integral equations involve the trace or normal derivative trace of the incident wavefield on $\Gamma$.
We  have already introduced the infinite vector $\UUt$ of the coefficients of $\uincg$ in the Fourier basis.
We then define similarly the infinite vector $\dUUt = (\dUUt^{p})_{1\leqslant p \leqslant M}$ of the coefficients of the normal derivative trace
 $\duincg$, such that
$$
\forall p=1,\ldots,M,\;\forall m\in\Zb,\qquad \dUUtmp = \PSGamma{\duincg}{\Phimp}.
$$
Finally, the density changes according to the integral equation and most particularly with respect to the boundary condition. 
To keep the same notations as previously, we introduce the densities
 $\lambda$ and $\psi$ (used in the BWIE) that are expanded in the Fourier basis as
$$
\lambda = \sum_{p=1}^{M}\sum_{m\in\Zb} \lambdamp\Phimp \qquad \text{ and }\qquad \psi = \sum_{p=1}^{M}\sum_{m\in\Zb} \psimp\Phimp.
$$
Finally, we set: $\lambdabt = (\lambdabt^{p})_{1\leqslant p \leqslant M}$ and
 $\Psit = (\Psit^{p})_{1\leqslant p \leqslant M}$, where each block  $\lambdabt^{p} = (\lambdabt_{m}^{p})_{m\in\Zb}$ and
  $\Psit^{p} = (\Psit_{m}^{p})_{m\in\Zb}$ is defined by: $\forall m\in\Zb$,
$\lambdabt_{m}^{p} = \lambdamp$ and $\Psit_{m}^{p} = \psimp$.



\subsection{Single-scattering preconditioned integral equations}

The EFIE preconditioned by its single scattering component (see Section \ref{??}), given by,
$$
\Lsgl^{-1} L \rho = \Lsgl^{-1}\uinc|_{\Gamma},
$$
can also be computed analytically in the Fourier bases. Indeed, let $\widehat{\Lb}^{-1}\Lb$ be the matrix associated to the operator $\Lsgl^{-1}L$, then
$$
\forall p,q=1,\ldots,\Nscat, \qquad
(\widehat{\Lb}^{-1}\Lb)^{p,q} = \begin{cases}
\Ib^{p} & \text{ if } p=q,\\
\dsp \sqrt{\frac{a_q}{a_p}}(\Hb^p)^{-1}(\Sbpq)^{T}\Jb^{q}&\text{otherwise}.
\end{cases}
$$
And for the Neumann case, the preconditioned integral equation (EFIE) reads as:
$$
\widehat{D}^{-1}D\lambda = \widehat{D}^{-1}\dn\uinc|_{\Gamma}.
$$ 
The matrix $\widehat{\Db}^{-1}\Db$ of $\widehat{D}^{-1}D$ is then given by
$$
\forall p,q=1,\ldots,\Nscat, \qquad
(\widehat{\Db}^{-1}\Db)^{p,q} = \begin{cases}
\Ib^{p} & \text{ if } p=q,\\
-\dsp \sqrt{\frac{a_q}{a_p}}(\dHbtp)^{-1}(\Sbpq)^{T}\dJb^{q}&\text{otherwise}.
\end{cases}
$$
As highlighted in Section \ref{??}, there is no need to compute the preconditioned versions of the other integral equations as they lead to the same operator (up to an invertible operators, for BWIE). To solve sound-hard or sound-soft scattering, using the above integral equations is an excellent choice.

%=================================================================
        \subsection{Projection of the incident waves in the Fourier basis}\label{secEqInt:SecondMembre}
%=================================================================

To fully solve one of the integral equations (EFIE, MFIE, CFIE or BWIE), we need to compute the Fourier coefficients of the
trace and normal derivative traces of the incident wave. We give the results  for both an incident plane wave 
and a point wise source term (Green's function).

For an incident plane wave, the following proposition holds \cite{AntChnRam08}.
\begin{prop}\label{propEqInt:UincFourier}
Let us assume that $\uinc$ is an incident plane wave of direction $\Beta$, with $\Beta = (\cos(\beta),\sin(\beta))$ and
 $\beta\in[0,2\pi]$, i.e.
$$
\forall\xx\in\Rb^{2},\qquad\uinc(\xx) = e^{ik\Beta\cdot\xx}.
$$
Then we have the following equalities
$$%\begin{equation}\label{eqEqInt:Fmp}
\UUt_{m}^{p}= \PSGamma{\uincg}{\Phimp} = \dmp \Jm(ka_p), \qquad  \dUUtmp=\PSGamma{\dn\uincg}{\Phimp} = k \dmp \Jmp(ka_p),
$$%\end{equation}
with
$%\begin{equation}\label{eqEqInt:dmp}
\dmp = \sqrt{2\pi a_p} e^{ik \Beta \cdot \bbp} e^{i m (\pi/2 - \beta)}
$.%\end{equation}
\end{prop}

Let us consider now an incident wave emitted by a pointwise source located at $\ssb\in\Omegaps$, i.e.
the wave $\uinc$ is the Green's function centered at $\ssb$.
The  Fourier coefficients  of the trace and normal derivative trace of $\uinc$ on $\Gamma$  are then given
by the following proposition \cite{ThierryThesis}.
\begin{prop}\label{propEqInt:UincGreenFourier}
Let $\ssb\in\Omegaps$. We assume that the incident wave $\uinc$ is the Green's function centered at $\ssb$ 
$$
\forall \xx\in\Rb^{2}\setminus\{\ssb\},\qquad \uinc(\xx) = G(\xx,\ssb) = \frac{i}{4}\Hz(k\|\xx-\ssb\|).
$$
The Fourier coefficients in $\Bscr$ of the trace and normal derivative trace of the incident wave on $\Gamma$ are respectively given by
$$%\begin{equation}\label{eqEqInt:FmpGreen}
\UUt_{m}^{p} = \PSGamma{\uincg}{\Phimp} = \frac{i \pi\ap}{2}\Jm(ka_p)\Hm(k\rp(\ssb))\overline{\Phimpt(\ssb)}
$$%\end{equation}
and
$$
\dUUtmp = \PSGamma{\dn\uincg}{\Phimp} = k \frac{i \pi\ap}{2} \Jmp(k \ap) H_{m}^{(1)}(k \rp(\ssb)) \overline{\Phimpt(\ssb)}.
$$
\end{prop}


%==============================================================================
%	\subsection{Near- and far-fields evaluations}\label{secEqInt:quantite}
%==============================================================================

%==============================================================================
	\subsection{Near-field evaluation}\label{secEqInt:evaluation}
%==============================================================================

	\subsubsection{Outside the obstacles}

By using the Graf's addition theorem \cite{Mar06,ThierryThesis}, we can compute the expression of the single- and double-layer potentials
at a point  $\xx$ located in the propagation domain $\Omegaps$:
\begin{prop}\label{propEqInt:LMphi}%[Evaluation des potentiels � l'ext�rieur des obstacles]
Let $\rho\in L^{2}(\Gamma)$ and $\mu\in H^{1/2}(\Gamma)$ be two densities admitting the following decompositions in the Fourier basis $\BF$ 
$$
\rho = \sum_{p=1}^{M}\sum_{m\in\Zb} \rhomp\Phimp \qquad\text{ and }\qquad \lambda = \sum_{p=1}^{M}\sum_{m\in\Zb} \lambdamp\Phimp.
$$
Then, for any point $\xx$ in the  domain of propagation $\Omegaps$, the single-layer potential reads
$$%\begin{equation}\label{eqEqInt:Lphimpt}
\Lop\rho(\xx) = \sum_{p=1}^M\sum_{m\in\Zb}\rhomp \Lop \Phimp(\xx) =\sum_{p=1}^M\sum_{m\in\Zb}\rhomp \frac{i\pi a_p}{2} J_m(ka_p)\Hm(kr_p(\xx)) \Phimpt(\xx),
$$%\end{equation}
and the double-layer potential can be expressed as
$$%\begin{equation}\label{eqEqInt:Mphimpt}
\Mop\lambda(\xx) = \sum_{p=1}^M\sum_{m\in\Zb}\lambdamp\Mop \Phimp(\xx) = -\sum_{p=1}^M\sum_{m\in\Zb} \lambdamp\frac{i\pi ka_p}{2} J_m'(ka_p)\Hm(kr_p(\xx)) \Phimpt(\xx).
$$%\end{equation}
\end{prop}
Proposition \ref{propEqInt:LMphi} implies that, for any  $\xx$ in $\Omegaps$,
$$
 u(\xx) = \Lop\rho(\xx) + \Mop\lambda(\xx) = \sum_{p=1}^M\sum_{m\in\Zb}\frac{i\pi a_p}{2} \left[\rhomp  J_m(ka_p) + \lambdamp \Jmp(k\ap)\right]\Hm(kr_p(\xx)) \Phimpt(\xx).
$$

	\subsubsection{Inside the obstacles}

In the same way, the potentials can be computed inside the obstacles, which is useful for penetrable obstacles for instance. In that case however, only the contribution of the current obstacle is taken into account:
$$
\um(\xx) = \Lop_p\rho_p + \Mop_p\lambda_p,\qquad\forall\xx\in\Omega_p.
$$

\begin{prop}\label{propEqInt:Inside}
Let $\rho\in L^{2}(\Gamma)$ and $\mu\in H^{1/2}(\Gamma)$ be two densities admitting the following decompositions in the Fourier basis $\BF$ 
$$
\rho = \sum_{p=1}^{M}\sum_{m\in\Zb} \rhomp\Phimp \qquad\text{ and }\qquad \lambda = \sum_{p=1}^{M}\sum_{m\in\Zb} \lambdamp\Phimp.
$$
Then, for any point $\xx$ inside the obstacle $\Omegap$, the single-layer potential reads
$$
\Lop\rho(\xx) = \sum_{m\in\Zb}\rhomp \Lop \Phimp(\xx) =\sum_{p=1}^M\sum_{m\in\Zb}\rhomp \frac{i\pi a_p}{2} H_m^{(1)}(ka_p)J_m(kr_p(\xx)) \Phimpt(\xx),
$$
and the double-layer potential can be expressed as
$$
\Mop\lambda(\xx) = \sum_{p=1}^M\sum_{m\in\Zb}\lambdamp\Mop \Phimp(\xx) = -\sum_{p=1}^M\sum_{m\in\Zb} \lambdamp\frac{i\pi ka_p}{2} \Hmp(ka_p)J_m(kr_p(\xx)) \Phimpt(\xx).
$$
\end{prop}


%==============================================================================
	\subsection{Far-field and Radar Cross Section (RCS)}\label{secEqInt:SER}
%==============================================================================

For computing the far-field pattern, let us recall that the scattered field $u$ admits the following Helmholtz's integral representation:
$u = \Lop \rho + \Mop \lambda$,
where $\rho$ and $\lambda$ are two unknown densities. In the polar coordinates system $(r,\theta)$ and by using an asymptotic expansion
of  $u$ when $r\to +\infty$, the following relation holds \cite{ColKre83}
$$
\forall \theta\in [0,2\pi],\qquad u(r,\theta) = \frac{e^{ikr}}{r^{1/2}} \left[ a_{\Lop}(\theta) + a_{\Mop}(\theta) \right] + \GrandO{\frac{1}{r^{3/2}}},
$$
where $a_{\Lop}$ and $a_{\Mop}$ are the radiated far-fields  for the single- and double-layer potentials, respectively, defined for any angle
 $\theta$ of $[0,2\pi]$ by
$$
\begin{cases}
\dsp{a_{\Lop}(\theta) = \frac{1}{\sqrt{8k\pi}}e^{i\pi /4} \int_{\Gamma} e^{-ik \thetab \cdot \yy} \rho(\yy) \dd \Gamma(\yy),}\\[0.4cm]
\dsp{a_{\Mop}(\theta) = \frac{1}{\sqrt{8k\pi}}e^{i\pi /4} \int_{\Gamma} -\frac{ik}{\|\yy\|} \thetab \cdot \yy e^{-ik \thetab \cdot \yy} \lambda(\yy) \dd \Gamma(\yy),}
\end{cases}
$$
with $\thetab:=(\cos(\theta),\sin(\theta))$. In addition, the Radar Cross Section (RCS) is defined by
$$%\begin{equation}\label{eqEqInt:defSER}
\forall \theta\in[0,2\pi], \quad \textrm{RCS}(\theta) = 10\log_{10}\left(2\pi\Abs{\aLop(\theta) + \aMop(\theta)}^{2}\right) (\dB).
$$%\end{equation}
To optimize the far-fields computation, these relations can be written thanks to the inner product between two infinite vectors.
Indeed, let us introduce $\aabt_{\Lop} = ((\aabt_{\Lop})^{p})_{1\leqslant p \leqslant M}$ and $\aabt_{\Mop} = ((\aabt_{\Mop})^{p})_{1\leqslant p \leqslant M}$,
where $(\aabt_{\Lop})^{p}$ and $(\aabt_{\Mop})^{p}$ are given by:  $\forall p=1,\ldots,M$,
$$
\left\{\begin{array}{l}
\dsp{(\aabt_{\Lop})^{p} = \Big((\aabt_{\Lop})^{p}_{m}\Big)_{m\in\Zb}, \qquad (\aabt_{\Lop})^{p}_{m} = \frac{ie^{-i\pi/4}\sqrt{a_{p}}}{2\sqrt{k}} e^{-i\bp k\cos(\theta-\alphap)}\Jm(k\ap)e^{im(\theta-\pi/2)},}\\[0.3cm]
\dsp{(\aabt_{\Mop})^{p} = \Big((\aabt_{\Lop})^{p}_{m}\Big)_{m\in\Zb}, \qquad (\aabt_{\Lop})^{p}_{m} = \frac{ie^{-i\pi/4}\sqrt{ka_{p}}}{2} e^{-i\bp k\cos(\theta-\alphap)}\Jmp(k\ap)e^{im(\theta-\pi/2)}.}
\end{array}\right.
$$
Then, we obtain the following: $\dsp{a_{\Lop}(\theta) = (\aabt_{\Lop})^{T}\Rhot}$ and $\dsp{a_{\Mop}(\theta) = (\aabt_{\Mop})^{T}\lambdabt}$.


%----------------------------------------
	\section{Finite-dimensional approximations and numerical solutions proposed in $\mu$-diff}\label{sectionFiniteDimensional}
%----------------------------------------

We now have all the ingredients to numerically solve the four integral equations EFIE, MFIE, CFIE and BWIE,
for sound-soft obstacles. In fact, any integral equation for any boundary condition can be solved according to the previous developments.
In practice, the infinite Fourier systems need to be truncated to get a finite dimensional problem: we must pass from
a sum over  $m\in \mathbb{Z}$ to a finite number of Fourier modes that depends on $ka_{p}$, $p=1,...,M$.
Let us consider e.g. the EFIE, the extension to the other boundary integral operators being direct. The EFIE is given by
 equation (\ref{eqEqInt:MatLbt}): $\Lbt\Rhot = -\UUt$. To truncate each Fourier series associated with 
  $(\Phimp)_{m\in\Zb}$ for the obstacle $\Omegamp$, we only keep $2N_{p} +1$ modes in such a way that the indices $m$ of the truncated series
  satisfy: $\forall p=1,\ldots,M$, $-\Np\leqslant m \leqslant \Np$. The truncation parameter $N_{p}$ must be fixed large enough, with
  $N_{p} \geqslant k a_{p}$, for $p=1,...,M$. An example \cite{AntChnRam08,JACT} is: $N_{p} = k a_{p} +�C_{p}$, where
   $C_{p}$ weakly grows with $k a_{p}$. A numerical study of the parameter $\Np$ is proposed in \cite{AntChnRam08,JACT} where 
   the following formula leads to a stable and accurate computation
\begin{equation}\label{eq:Np}
 \Np = \left[ k a_p+ \left(\frac{1}{2\sqrt{2}} \ln (2\sqrt{2} \pi k a_p \varepsilon^{-1})\right)^{\frac{2}{3}} (k a_p)^{1/3} +1\right ],
\end{equation}
where $\varepsilon$ is a small parameter (related to the relative tolerance required in the iterative Krylov subspace solver used
for solving the truncated linear system (\ref{EFIEDimensionFinie}), see  \cite{AntChnRam08,JACT}).

The resulting  linear system writes
\begin{equation}\label{EFIEDimensionFinie}
\Lb\Rho = -\UU,
\end{equation}
where we introduced the block matrix $\Lb = (\Lbpq)_{1\leqslant p,q\leqslant M}$ and the vectors $\Rho = (\Rho^{p})_{1\leqslant p\leqslant M}$ and
 $\UU = (\UU^{p})_{1\leqslant p\leqslant M}$ defined by
\begin{equation}\label{eqEqInt:EFIEtronque}
\Lb = \left[
\begin{array}{c c c c}
\Lb^{1,1} & \Lb^{1,2} & \ldots & \Lb^{1,M} \\
\Lb^{2,1} & \Lb^{2,2} & \ldots & \Lb^{2,M} \\
\vdots & \vdots & \ddots & \vdots\\
\Lb^{M,1} & \Lb^{M,2} & \ldots & \Lb^{M,M}
\end{array}
\right],
\qquad \qquad
\Rho =
\left[
\begin{array}{c}
\Rho^{1} \\
\Rho^{2} \\
\vdots \\
\Rho^{M}
\end{array}
\right],\qquad \qquad
\UU =
\left[
\begin{array}{c}
\UU^{1} \\
\UU^{2} \\
\vdots \\
\UU^{M}
\end{array}
\right].
\end{equation}
For $p,q= 1,\ldots, M$,
 the  complex-valued matrix $\Lbpq$ is of size $(2\Np+1)\times(2N_{q}+1)$ and its coefficients $\Lbpqmn$ are: $\Lbpqmn = \Lbtpqmn$,
 for $m=-\Np,\ldots,\Np$, $ n = -\Nq,\ldots,\Nq$.
The complex-valued components of the vector $\Rho^{p} = (\Rho^{p}_{m})_{-\Np\leqslant m \leqslant\Np}$ of size $2\Np+1$ are
  the approximate Fourier coefficients  $\rhomp$ of $\rho$. For the sake of clarity, we keep on writing: $\Rho^{p}_{m} = \Rhot^{p}_{m} = \rhomp$,
  for all $ m =-\Np,\ldots,\Np$.
The complex-valued vector $\UU^{p} = (\UU^{p}_{m})_{-\Np\leqslant m \leqslant\Np}$ is composed of the  $2\Np+1$ Fourier
coefficients of the trace of the incident wave on $\Gamma$, i.e.  $\UU^{p}_{m} = \UUt^{p}_{m} = \PSGamma{\uincg}{\Phimp}$,
$\forall m =-\Np,\ldots,\Np$.
If $\Ntot = \sum_{p=1}^{M}(2\Np +1)$ denotes the total number of modes, the size of the complex-valued matrix
 $\Lb$ is then $\Ntot\times\Ntot$. More generally, all the boundary integral operators can be truncated according to this process.
 Concerning the notations, it is sufficient to formally omit the tilde symbol $\sim$ over the quantities involved in sections
  (\ref{integralinfinite}) to (\ref{secEqInt:SER}).

Since  the four finite-dimensional matrices $\Lb$, $\Mb$, $\Nb$ and $\Db$ that respectively correspond to the four
boundary integral operators $L$, $M$, $N$ and $D$ can be computed,  the linear systems that approximate the EFIE, MFIE, CFIE and
 BWIE can be stated. For example, the CFIE leads to (with $0\leqslant \alpha\leqslant 1$ and $\Im(\eta)\neq 0$)
\begin{equation}\label{eqEqInt:CFIEDapp}
\left[ \alpha\eta\Lb + (1-\alpha) \left(\frac{\Ib}{2} + \Nb\right)\right]\Rho = -\alpha\eta\UU - (1-\alpha)\dUU.
\end{equation}
Let us remark that the matrix obtained after discretization is always a linear combination of the four integral operators
 $\Lb$, $\Mb$, $\Nb$, $\Db$ and the identity matrix  $\Ib$. As a consequence, for a given integral equation,
 the resulting matrix is of size $\Ntot\times\Ntot$ and has the same block structure as e.g. 
  $\Lb$ (see equation  (\ref{eqEqInt:EFIEtronque})).
 The finite-dimensional linear system (\ref{EFIEDimensionFinie}) (or (\ref{eqEqInt:CFIEDapp})) is accurately solved in $\mu$-diff
by using the Matlab direct solver or a preconditioned Krylov subspace linear solver 
that uses fast matrix-vector products based on Fast Fourier Transforms (FFTs), the choice of the linear algebra strategy
(direct vs. iterative) depending on the configuration with respect to $ka_{p}$ and  
$M$. The use of FFTs is made possible since the off-diagonal blocks of the integral operators can be written as the products
of diagonal and Toeplitz matrices \cite{AntChnRam08,JACT} (see e.g. the matrices $\Sbtpqmn$ in section \ref{integralinfinite}). In addition,
low memory is only necessary when $ka_{p}$ is large enough since  the storage of the Toeplitz
matrices can be optimized. This resulting storage technique is called \textit{sparse} representation in $\mu$-diff,
 in contrast with the \textit{dense} (full) storage of the complex-valued matrices.
 Let us assume that $a_p \approx a$, for $1 \leqslant p \leqslant M$.
In terms of storage, the dense version of a matrix requires to store about $4M^{2}[ka]^{2}$ coefficients (assuming that $N_{p}$ are fixed
by formula (\ref{eq:Np}), and $[r]$ denotes the integer part of a real number $r$) while the sparse storage needs about
$4M^{2}[ka]$ complex-valued coefficients. In terms of computational time for solving the linear system,
the direct (multithreaded) gaussian solver included in Matlab leads to a cost that scales with $\mathcal{O}(M^{3}(ka)^{3})$. For the preconditioned iterative Krylov subspace
methods (i.e. restarted GMRES)), the global cost is $\mathcal{O}(M^2 ka \log_{2}(ka))$, the converge rate depending on
the physical situation and robustness of the preconditioner. From these remarks, we deduce that an iterative method is an efficient  and cheap 
alternative to a direct solver for large wavenumbers $ka$, but also for large $M$.
 We refer to \cite{AntChnRam08,JACT} for a thorough computational study of the various numerical strategies.
A few examples in $\mu$-diff are provided with the toolbox.
Finally, the post-processing formulas (near- and far-fields quantities) clearly inherits of the truncation procedure (see sections \ref{secEqInt:evaluation} and \ref{secEqInt:SER}).
