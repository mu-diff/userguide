\documentclass[11pt,twoside,a4paper]{book}
% Pour avoir les lettres accentu�es  et l'hyph�nation
\usepackage[T1, T5]{fontenc}
\usepackage{lmodern}
\usepackage[applemac]{inputenc}
\usepackage[english]{babel}



\usepackage{epsfig}
\usepackage{mathrsfs}
\usepackage[usenames,dvipsnames]{color}
\usepackage{amsthm}
\usepackage{amsmath}
\usepackage{amsfonts}
\usepackage{amssymb}
%\usepackage{showkeys}
\usepackage{epstopdf}
\usepackage{graphicx}
\usepackage{graphics}
\usepackage{subfigure}

\usepackage{pgfplots}
\usepackage{pgfkeys}

\usepackage{import} % pour les .pdf_tex

\usepackage{pdfsync}


\usepackage{fullpage}
\usepackage{xspace}
%\usepackage[table]{xcolor}

\usepackage{minitoc} % pour la table des matieres � chaque chapitre


%%% FOOTNOTE
%\usepackage{footnpag}
\usepackage{footmisc}

%% Mes macros � moi
\usepackage{MesCommandes}

%\usepackage[framed,numbered,autolinebreaks,useliterate]{mcode}
\usepackage[framed,autolinebreaks,useliterate]{mcode}


%% HYPERREF (pdf interactif) A PLACER EN DERNIER !!!
\usepackage[pdftex]{hyperref}


%Inline code in a text ...
\newcommand{\code}[1]{\mcode{#1}}
%.... and in a tabular
\newcommand{\tabcode}[1]{\lstbasicfont{#1}}

\newcommand{\blockCode}[1]{\begin{lstlisting}{#1}\end{lstlisting}}

\newcommand{\folder}[1]{\texttt{#1}}

%\newcommand{\hilight}[1]{\colorbox{yellow}{#1}}
%\newcommand{\funByName}[3]{\textbullet\;\;\texttt{#1} (\textit{#2})\\#3\\[0.1cm]}

\usepackage[framemethod=tikz]{mdframed}
% To avoid vertical spacing... :
\parindent0mm
\let\svendmdframed\endmdframed
\makeatletter
\def\endmdframed{\svendmdframed\unskip\vspace{-6pt}}
\makeatother
%To describe a function
\definecolor{light-gray}{gray}{0.95}
\newcommand{\funByNameAux}[3]{\textbullet\;\;\texttt{#1} (\textit{#2})\\#3}
\newcommand{\funByName}[3]{\begin{mdframed}[hidealllines=true, backgroundcolor=blue!20, skipbelow=0pt, skipabove=0pt]\funByNameAux{#1}{#2}{#3}\end{mdframed}}
\newcommand{\funByNameEven}[3]{\begin{mdframed}[hidealllines=true, backgroundcolor=light-gray, skipbelow=2pt, skipabove=2pt]\funByNameAux{#1}{#2}{#3}\end{mdframed}}
\newcommand{\funByNameUneven}[3]{\begin{mdframed}[hidealllines=true, backgroundcolor=white, skipbelow=2pt, skipabove=2pt]\funByNameAux{#1}{#2}{#3}\end{mdframed}}


\newcounter{FolderCounter}
\setcounter{FolderCounter}{0}

\newcommand{\NewFolder}[1]{\stepcounter{FolderCounter}\medskip\textbf{\large #1}\\}

\newcommand{\funByFolderAux}[2]{\textbullet\;\;\texttt{#1}: #2}
\newcommand{\funByFolder}[2]{\funByFolderAux{#1}{#2}\\[0.1cm]}
\newcommand{\funByFolderEven}[2]{\begin{mdframed}[hidealllines=true, backgroundcolor=light-gray, skipbelow=2pt, skipabove=2pt]\funByFolderAux{#1}{#2}\end{mdframed}}
\newcommand{\funByFolderUneven}[2]{\begin{mdframed}[hidealllines=true, backgroundcolor=white, skipbelow=2pt, skipabove=2pt]\funByFolderAux{#1}{#2}\end{mdframed}}


%% For chain in Latex
\usepackage{tikz}
\usetikzlibrary{calc,trees,positioning,arrows,chains,shapes.geometric,%
    decorations.pathreplacing,decorations.pathmorphing,shapes,%
    matrix,shapes.symbols}

\tikzset{
>=stealth',
  punktchain/.style={
    rectangle, 
    rounded corners, 
    % fill=black!10,
    draw=black, very thick,
    text width=10em, 
    minimum height=3em, 
    text centered, 
    on chain},
  line/.style={draw, thick, <-},
  element/.style={
    tape,
    top color=white,
    bottom color=blue!50!black!60!,
    minimum width=8em,
    draw=blue!40!black!90, very thick,
    text width=10em, 
    minimum height=3.5em, 
    text centered, 
    on chain},
  every join/.style={->, thick,shorten >=1pt},
  decoration={brace},
  tuborg/.style={decorate},
  tubnode/.style={midway, right=2pt},
}
%%%% end chain

%MU DIFF FUNCTIONS
\usepackage{mudiff}


\title{\huge \mudiff Toolbox for Matlab: user guide}

%\author{X. Antoine\thanks{Institut Elie Cartan Nancy (IECN), Nancy University, INRIA Corida Team, B.P. 239, F-54506
%    Vandoeuvre-l\`es-Nancy Cedex, France.}, \and
%  C. Besse\thanks{?} \and P. Klein}

\author{\Large Xavier \textsc{Antoine} and Bertrand \textsc{Thierry}}

\date{{\large \today}\\[0.5cm]version $0.1$}

\begin{document}

%\maketitle

\renewcommand{\thefootnote}{\fnsymbol{footnote}}

%\footnotetext[1]{Laboratoire J.L. Lions (LJLL), University of Paris VI, Paris, France. ({\tt bertrand.thierry1@gmail.com}).}
%\footnotetext[2]{Universit\'e de Lorraine, Institut Elie Cartan de Lorraine, UMR 7502, Vandoeuvre-l\`es-Nancy, F-54506, France. ({\tt xavier.antoine@univ-lorraine.fr}).}
%\footnotetext[3]{Department of Mathematics, University College in Qunfudah, Umm Al-Qura University, Saudi Arabia.
% ({\tt cachniti@uqu.edu.sa;hmzubaidi@uqu.edu.sa}).}


\renewcommand{\thefootnote}{\arabic{footnote}}


\maketitle
\dominitoc
\tableofcontents


\chapter*{Copyright}
\mtcaddchapter[Copyright]                          % solution pour minitoc
\markboth{\uppercase{Copyright}}{\uppercase{Copyright}} 

\ldots 

\chapter*{Introduction}
\mtcaddchapter[Introduction]                          % solution pour minitoc
\markboth{\uppercase{Introduction}}{\uppercase{Introduction}} 

\ldots 

\chapter*{Installation}
\input{Install}


\chapter{Boundary integral equations}\minitoc
\label{chap:math}
The \mudiff toolbox is based on integral equation and, in fact, provides the the four classical boundary integral operators. The derivation and the choice of the integral formulation is let to the user, even if some of them are given in this chapter, which contains all the necessary mathematical background to solve  time-harmonic wave scattering problems by disks, by penetrable or impenetrable obstacles. The only limitation is the integral formulation: if the integral formulation is written, then it can be solved using \mudiff.

This chapter begins by presenting the potential theory and the four classical boundary integral operators with their main properties. The case of the scattering by disks is then studied and the boundary integral operators are projected in the Fourier bases, leading to infinite matrices but with analytic expression of the coefficients. The case of the truncation is then discussed, and the chapter concludes with the expression of both the near- and far-field, and the projections of the right-hand side on the Fourier bases (incident wave).

\chapter{Multiple scattering by disks}\minitoc
\label{chap:fourier}
\section{Spectral formulation used in $\mu$-diff}\label{MuDiffFormulation}
\label{secEqInt:disque}

We consider now  circular cylinders as scatterers. In this situation, we can explicitly compute  the boundary integral
equations in a Fourier basis,  leading therefore to an efficient computational spectral method when used in conjunction
with numerical linear algebra methods (direct or iterative solvers).
%====================================
        \subsection{Notations and Fourier basis}\label{secEqInt:BaseFourier}
%====================================

Let us consider an orthonormal system $(\OO,\V{\OO x_{1}},\V{\OO x_{2}})$. We assume that the scattering obstacle
 $\Omegam$ is the union of  $M$ disks $\Omegamp$, for $p = 1,\ldots,M$, of radius $a_p$ and center $\OOp$.
 We define $\Gamma_p$ as the boundary of  $\Omegamp$ and by $\dsp{\Gamma = \cup_{p=1 \ldots M}\Gamma_p}$ the boundary of
  $\Omegam$.  The unit normal vector $\nn$ to $\Omegam$ is outgoing. An illustration of the notations is reported on
  Figure \ref{figEqInt:schemanotations}.
  
  \begin{figure}[h!]
\begin{center}
\def\svgwidth{10cm}
\import{./img/ChapEqInt/disques/}{schema.pdf_tex}
\end{center}
\caption{Illustration of the notations for two disks $\Omegamp$ and $\Omegamq$ and a point $\xx�\in \Omegaps$.}
\label{figEqInt:schemanotations}
\end{figure}
  
For any $p=1, \ldots, M$, we introduce $\bbp $ as the vector between the center $\OOp$ and the origin $\OO$
$$%\begin{equation}\label{eqEqInt:bp}
\bbp = \OO \OOp, \qquad \qquad b_p= \left\|\bbp\right\|, \qquad\qquad \alpha_p=Angle(\V{\OO x_{1}}, \bbp),
$$%\end{equation}
and, for $q = 1,\ldots,M$, with $q \neq p$, $\bbpq$ as the vector between the centers $\OOq$ and $\OOp$ 
$$%\begin{equation}\label{eqEqInt:bpq}
\bbpq = \OOq\OOp, \qquad \qquad b_{pq}=\left\|\bbpq  \right\|, \qquad\qquad
\alpha_{pq}=Angle(\V{\OO x_{1}},\bbpq ).
$$%\end{equation}
Furthermore, any point $\xx$ is described by its global polar coordinates
$$
\rr (\xx)= \OO \xx,  \qquad \qquad r (\xx)=\left\|\rr(\xx) \right\|,  \qquad\qquad
\theta(\xx)=Angle(\V{\OO x_{1}},\br(\xx)),
$$
or by its polar coordinates in the orthonormal system associated with the obstacle $\Omegamp$, with $p =1, \ldots, M$, 
$$%\begin{equation}\label{eqEqInt:PolaireLocal}
\rrp (\xx)= \OOp \xx, \qquad \qquad r_p(\xx)  = \left\|\rrp(\xx)\right\|,  \qquad
\qquad \theta_p(\xx) = Angle(\V{\OOp x_{1}}, \rrp(\xx)).
$$%\end{equation}


Let us now build a basis of  $L^2(\Gamma)$ to approximate the integral operators. To this end, we first construct
 a basis of $L^2(\Gamma_p)$ associated with $\Omegamp$, for
 $p=1,\ldots,M$. If the circle $\Gamma_p$ has a radius one and is centered
 at the origin, then a suitable basis of $L^{2}(\Gammap)$ is the spectral Fourier basis of functions $(e^{im\theta})_{m \in \Zb}$.
 We adapt this basis to the general case where $\ap \neq 1$ by introducing, on one hand, the functions $(\phim)_{m\in\Zb}$ defined on $\Rb^{2}$ by: 
$\forall m\in\Zb$, $\forall \xx \in \Rbb$, $\varphi_m(\xx) = e^{im\theta(\xx)}$,
and, on the other hand,  the functions  $(\phimp)_{1\leqslant p \leqslant M, \; m\in\Zb}$  given by
$$%\begin{equation}\label{eqEqInt:phimp}
\forall p=1,\ldots,M, \forall m\in\Zb, 
\forall \xx \in \Gamma_p, \qquad\varphi_m^p(\xx) = \frac{\varphi_m(\rr_p(\xx))}{\sqrt{2 \pi a_p}} = \frac{e^{im\theta_p(\xx)}}{\sqrt{2 \pi a_p}}.
$$%\end{equation}
For $p=1,\ldots,M$, the family $\dsp{\left( \varphi_m^p \right)_{m \in \Zb}}$ forms an orthonormal basis of
 $L^2(\Gamma_p)$ for the hermitian inner product $\PSGammap{\cdot}{\cdot}$
$$
\forall f,g\in L^{2}(\Gammap),\qquad \PSGammap{f}{g} = \int_{\Gammap} f(\xx) \overline{g(\xx)} \dd\Gammap(\xx).
$$
To build a basis of $L^2(\Gamma)$, we introduce the functions $\Phi_m^p$ of $L^2(\Gamma)$ as the union of these $M$ families 
$$%\begin{equation}\label{eqEqInt:Phimp}
\forall p,q = 1,\ldots, M, \forall m\in\Zb, \qquad \Phi_m^p |_{\Gamma_q} = 
\begin{cases}
        0 & \text{if } q \neq p, \\
        \varphi_m^p & \text{if } q = p.
\end{cases}
$$%\end{equation}
The family $\BF = \{ \Phi_m^p, \, m \in \Zb, p =1, \ldots, M\}$, also called Fourier or spectral basis, is a Hilbert basis of
 $L^2(\Gamma)$ for the usual scalar product $(\cdot,\cdot)_{L^{2}(\Gamma)}$. 


\subsection{Integral operators - integral equations for a cluster of circular cylinders}\label{integralinfinite}


In view of a numerical procedure, $\mu$-diff uses the weak formulation of the EFIE (\ref{EFIED}) in 
 $L^{2}(\Gamma)$ based on the Fourier basis $\BF$ 
$$
\begin{cases}
\text{Find $\rho \in H^{-1/2}(\Gamma)$ such that for any $p=1,\ldots,M,$ and $m\in\Zb$,} & \\
\PSGamma{L\rho}{\Phimp} = -\PSGamma{\uincg}{\Phimp}. &
\end{cases}
$$
Since $\uinc$ is assumed to be smooth enough (typically $\Cscr^{\infty}$) and that $\Gamma$ is $\Cscr^{\infty}$, then
the scattered wavefield is also $\Cscr^{\infty}(\Omegaps)$ and the density  $\rho$ is (at least) in
 $H^{1/2}(\Gamma)$. Therefore, $\rho$ can be expanded in  $\BF$ as 
$$
\rho = \sum_{q=1}^{M}\sum_{n\in\Zb}\rhonq\Phinq$$ and the weak form of the EFIE  is 
$$
\begin{cases}
\text{Find the Fourier coefficients $\rhonq\in\Cb$, for $q=1,\ldots,M$, and $n\in\Zb$, such that,} & \\
\dsp{\forall p=1,\ldots,M, \;\forall m\in\Zb, \qquad \sum_{q=1}^{M}\sum_{n\in\Zb}\rhonq\PSGamma{L\Phinq}{\Phimp} = -\PSGamma{\uincg}{\Phimp}}. &
\end{cases}
$$
This formulation can be written under the following matrix form $
\Lbt\Rhot = \UUt$,
where the infinite matrix representation
 $\Lbt =(\Lbtpq)_{1\leqslant p,q\leqslant M}$ and the infinite vectors
  $\Rhot=(\Rhot^{p})_{1\leqslant p\leqslant M}$ and
   $\UUt=(\UUt^{p})_{1\leqslant p\leqslant M}$ are defined by blocks   as
\begin{equation}\label{eqEqInt:MatLbt}
\Lbt =
\left[
\begin{array}{c c c c}
\Lbt^{1,1} & \Lbt^{1,2} & \ldots & \Lbt^{1,M} \\
\Lbt^{2,1} & \Lbt^{2,2} & \ldots & \Lbt^{2,M} \\
\vdots & \vdots & \ddots & \vdots\\
\Lbt^{M,1} & \Lbt^{M,2} & \ldots & \Lbt^{M,M}
\end{array}
\right],
\qquad \qquad
\Rhot =
\left[
\begin{array}{c}
\Rhot^{1} \\
\Rhot^{2} \\
\vdots \\
\Rhot^{M}
\end{array}
\right],\qquad 
\UUt =
\left[
\begin{array}{c}
\UUt^{1} \\
\UUt^{2} \\
\vdots \\
\UUt^{M}
\end{array}
\right],
\end{equation}
with, for any $p,q=1,\ldots,M$, and $m,n\in\Zb$:
$
\Lbtpqmn = \PSGamma{L\Phinq}{\Phimp}$, $\Rhot_{m}^{p} = \rhomp$ and
$\UUt_{m}^{p} = \PSGamma{-\uincg}{\Phimp}$.

For the other integral formulations (section \ref{AutresEI}) or even for any other boundary condition, 
 the expressions of the three boundary integral operators  $M$, $N$ and $D$ are needed.
Therefore, to compute an integral equation, we introduce the infinite matrices $\Mbt = (\Mbtpq)_{1\leqslant p,q\leqslant M}$, $\Nbt= (\Nbtpq)_{1\leqslant p,q\leqslant M}$ and
 $\Dbt= (\Dbtpq)_{1\leqslant p,q\leqslant M}$, with the same block structure as $\Lbt$ (see equation
  (\ref{eqEqInt:MatLbt})). For $p,q =1,\ldots, M$, the coefficients of the infinite matrices $\Mbtpq$, $\Nbtpq$ and $\Dbtpq$ are defined for any indices
   $m$ and $n$ in $\Zb$ by
\[
\Mbtpqmn = \PSGamma{M\Phinq}{\Phimp},
\Nbtpqmn = \PSGamma{N\Phinq}{\Phimp}, \textrm{ and }
\Dbtpqmn = \PSGamma{D\Phinq}{\Phimp}.
\]
For a numerical implementation, we can explicitly compute \cite{JACT,ThierryThesis} the matrix blocks
 $\Lbtpq$, $\Mbtpq$, $\Nbtpq$ and $\Dbtpq$ involved in $\Lbt$, $\Mbt$, $\Nbt$ and $\Dbt$, for
  $p,q=1,\ldots,M$. To this end, we introduce the infinite diagonal matrices
   $\Jbtp$, $\dJbtp$, $\Hbtp$ and $\dHbtp$, with general terms, for $m \in \Zb$,
$$%\begin{equation}\label{eqEqInt:Jbtp}
\Jbtp_{mm} = \Jm(ka_{p}), \hspace{0.5cm}
\dJbtp_{mm} = \Jmp(ka_{p}), \hspace{0.5cm}
\Hbtp_{mm} = \Hm(ka_{p}), \hspace{0.5cm}
\dHbtp_{mm} = \Hmp(ka_{p}).
$$%\end{equation}
In addition, let $\Ibtp$ be the infinite identity matrix, and, for $q\neq p$, the infinite separation matrix $\Sbtpq$ between the obstacles $\Omegamp$ and
 $\Omegamq$, defined by
$$%\begin{equation}\label{eqEqInt:Sbtpq}
\Sbtpq=(\Sbtpqmn)_{m\in \Zb,n\in
\Zb}\qquad \text{and}\qquad \Sbtpqmn=\Smnbpq = H_{m-n}^{(1)} (k b_{pq})e^{i(m-n)\alpha_{bq}}.
$$%\end{equation}
Under these notations, we rewrite the blocks $\Lbtpq$, $\Mbtpq$, $\Nbtpq$ and $\Dbtpq$ of the infinite matrices
 $\Lbt$, $\Mbt$, $\Nbt$ and $\Dbt$ under the matrix form, for any $p,q= 1,\ldots,M$,
 \begin{itemize}
\item[]\begin{flalign}\label{eq:InfL}
\text{\textbullet}\quad&
\Lbtpq = 
\begin{cases}
\dsp{\frac{i\pi a_p}{2} \Jbtp\Hbtp,} & \text{ if } p = q,\\[0.3cm]
\dsp{\frac{i \pi\sqrt{a_p a_q}}{2} \Jbtp (\Sbtpq)^T\Jbtq,} &\text{ if } p \neq q,
\end{cases}&\end{flalign}\item[]\begin{flalign}\label{eq:InfM}
\text{\textbullet}\quad&
\Mbtpq = 
\begin{cases}
\dsp{ - \frac{1}{2}\Ibtp - \frac{i \pi k a_p}{2} \Jbtp \dHbtp = \frac{1}{2}\Ibtp - \frac{i \pi k a_p}{2} \dJbtp \Hbtp,} &\text{ if } p = q,\\[0.3cm]
\dsp{- \frac{i k \pi\sqrt{a_p a_q}}{2} \Jbtp (\Sbtpq)^T \dJbtq,}&\text{ if } p \neq q,
\end{cases}&\end{flalign}\item[]\begin{flalign}\label{eq:InfN}
\text{\textbullet}\quad&
\Nbtpq =
\begin{cases}
\dsp{\frac{1}{2}\Ibtp + \frac{i \pi k a_p}{2} \Jbtp \dHbtp  = -\frac{1}{2}\Ibtp + \frac{i \pi k a_p}{2} \dJbtp \Hbtp,} &\text{ if } p = q, \\[0.3cm]
\dsp{\frac{i k \pi\sqrt{a_p a_q}}{2} \dJbtp (\Sbtpq)^T \Jbtq,}& \text{ if } p \neq q, 
\end{cases}&\end{flalign}\item[]\begin{flalign}\label{eq:InfD}
\text{\textbullet}\quad&
\Dbtpq =
\begin{cases}
\dsp{\frac{i \pi k^2 a_p}{2} \dJbtp\dHbtp,} & \text{ if } p = q, \\[0.3cm]
\dsp{ - \frac{i k^2 \pi\sqrt{a_p a_q}}{2} \dJbtp (\Sbtpq)^T \dJbtq,} &\text{ if } p \neq q,
\end{cases}&\end{flalign}\end{itemize} 
where $(\Sbtpq)^T$ is the transpose matrix of the separation matrix $\Sbtpq$. 

The integral equations involve the trace or normal derivative trace of the incident wavefield on $\Gamma$.
We  have already introduced the infinite vector $\UUt$ of the coefficients of $\uincg$ in the Fourier basis.
We then define similarly the infinite vector $\dUUt = (\dUUt^{p})_{1\leqslant p \leqslant M}$ of the coefficients of the normal derivative trace
 $\duincg$, such that
$$
\forall p=1,\ldots,M,\;\forall m\in\Zb,\qquad \dUUtmp = \PSGamma{\duincg}{\Phimp}.
$$
Finally, the density changes according to the integral equation and most particularly with respect to the boundary condition. 
To keep the same notations as previously, we introduce the densities
 $\lambda$ and $\psi$ (used in the BWIE) that are expanded in the Fourier basis as
$$
\lambda = \sum_{p=1}^{M}\sum_{m\in\Zb} \lambdamp\Phimp \qquad \text{ and }\qquad \psi = \sum_{p=1}^{M}\sum_{m\in\Zb} \psimp\Phimp.
$$
Finally, we set: $\lambdabt = (\lambdabt^{p})_{1\leqslant p \leqslant M}$ and
 $\Psit = (\Psit^{p})_{1\leqslant p \leqslant M}$, where each block  $\lambdabt^{p} = (\lambdabt_{m}^{p})_{m\in\Zb}$ and
  $\Psit^{p} = (\Psit_{m}^{p})_{m\in\Zb}$ is defined by: $\forall m\in\Zb$,
$\lambdabt_{m}^{p} = \lambdamp$ and $\Psit_{m}^{p} = \psimp$.



\subsection{Single-scattering preconditioned integral equations}

The EFIE preconditioned by its single scattering component (see Section \ref{sec:SingleScat}), given by,
$$
\Lsgl^{-1} L \rho = \Lsgl^{-1}\uinc|_{\Gamma},
$$
can also be computed analytically in the Fourier bases. Indeed, let $\widehat{\Lb}^{-1}\Lb$ be the matrix associated to the operator $\Lsgl^{-1}L$, then
\begin{equation}\label{eq:LL}
\forall p,q=1,\ldots,\Nscat, \qquad
(\widehat{\widetilde{\Lb}}^{-1}\widetilde{\Lb})^{p,q} = \begin{cases}
\Ib^{p} & \text{ if } p=q,\\
\dsp \sqrt{\frac{a_q}{a_p}}(\Hbt^p)^{-1}(\Sbtpq)^{T}\Jbt^{q}&\text{otherwise}.
\end{cases}
\end{equation}
And for the Neumann case, the preconditioned integral equation (EFIE) reads as:
$$
\widehat{D}^{-1}D\lambda = \widehat{D}^{-1}\dn\uinc|_{\Gamma}.
$$ 
The matrix $\widehat{\Db}^{-1}\Db$ of $\widehat{D}^{-1}D$ is then given by
\begin{equation}\label{eq:DD}
\forall p,q=1,\ldots,\Nscat, \qquad
(\widehat{\widetilde{\Db}}^{-1}\widetilde{\Db})^{p,q} = \begin{cases}
\Ib^{p} & \text{ if } p=q,\\
-\dsp \sqrt{\frac{a_q}{a_p}}(\dHbtp)^{-1}(\Sbtpq)^{T}\dJbt^{q}&\text{otherwise}.
\end{cases}
\end{equation}
As highlighted by Proposition \ref{prop:SingleScat}, there is no need to compute the preconditioned versions of the other integral equations as they lead to the same operator (up to an invertible operators, for BWIE). To solve sound-hard or sound-soft scattering, using the above integral equations is an excellent choice.

%=================================================================
        \subsection{Projection of the incident waves in the Fourier basis}\label{secEqInt:SecondMembre}
%=================================================================

To fully solve one of the integral equations (EFIE, MFIE, CFIE or BWIE), we need to compute the Fourier coefficients of the
trace and normal derivative traces of the incident wave. We give the results  for both an incident plane wave 
and a point wise source term (Green's function).

For an incident plane wave, the following proposition holds \cite{AntChnRam08}.
\begin{prop}\label{propEqInt:UincFourier}
Let us assume that $\uinc$ is an incident plane wave of direction $\Beta$, with $\Beta = (\cos(\beta),\sin(\beta))$ and
 $\beta\in[0,2\pi]$, i.e.
$$
\forall\xx\in\Rb^{2},\qquad\uinc(\xx) = e^{ik\Beta\cdot\xx}.
$$
Then we have the following equalities
$$%\begin{equation}\label{eqEqInt:Fmp}
\UUt_{m}^{p}= \PSGamma{\uincg}{\Phimp} = \dmp \Jm(ka_p), \qquad  \dUUtmp=\PSGamma{\dn\uincg}{\Phimp} = k \dmp \Jmp(ka_p),
$$%\end{equation}
with
$%\begin{equation}\label{eqEqInt:dmp}
\dmp = \sqrt{2\pi a_p} e^{ik \Beta \cdot \bbp} e^{i m (\pi/2 - \beta)}
$.%\end{equation}
\end{prop}

Let us consider now an incident wave emitted by a pointwise source located at $\ssb\in\Omegaps$, i.e.
the wave $\uinc$ is the Green's function centered at $\ssb$.
The  Fourier coefficients  of the trace and normal derivative trace of $\uinc$ on $\Gamma$  are then given
by the following proposition \cite{ThierryThesis}.
\begin{prop}\label{propEqInt:UincGreenFourier}
Let $\ssb\in\Omegaps$. We assume that the incident wave $\uinc$ is the Green's function centered at $\ssb$ 
$$
\forall \xx\in\Rb^{2}\setminus\{\ssb\},\qquad \uinc(\xx) = G(\xx,\ssb) = \frac{i}{4}\Hz(k\|\xx-\ssb\|).
$$
The Fourier coefficients in $\Bscr$ of the trace and normal derivative trace of the incident wave on $\Gamma$ are respectively given by
$$%\begin{equation}\label{eqEqInt:FmpGreen}
\UUt_{m}^{p} = \PSGamma{\uincg}{\Phimp} = \frac{i \pi\ap}{2}\Jm(ka_p)\Hm(k\rp(\ssb))\overline{\Phimpt(\ssb)}
$$%\end{equation}
and
$$
\dUUtmp = \PSGamma{\dn\uincg}{\Phimp} = k \frac{i \pi\ap}{2} \Jmp(k \ap) H_{m}^{(1)}(k \rp(\ssb)) \overline{\Phimpt(\ssb)}.
$$
\end{prop}


%==============================================================================
%	\subsection{Near- and far-fields evaluations}\label{secEqInt:quantite}
%==============================================================================

%==============================================================================
	\subsection{Near-field evaluation}\label{secEqInt:evaluation}
%==============================================================================

	\subsubsection{Outside the obstacles}

By using the Graf's addition theorem \cite{Mar06,ThierryThesis}, we can compute the expression of the single- and double-layer potentials
at a point  $\xx$ located in the propagation domain $\Omegaps$:
\begin{prop}\label{propEqInt:LMphi}%[Evaluation des potentiels � l'ext�rieur des obstacles]
Let $\rho\in L^{2}(\Gamma)$ and $\mu\in H^{1/2}(\Gamma)$ be two densities admitting the following decompositions in the Fourier basis $\BF$ 
$$
\rho = \sum_{p=1}^{M}\sum_{m\in\Zb} \rhomp\Phimp \qquad\text{ and }\qquad \lambda = \sum_{p=1}^{M}\sum_{m\in\Zb} \lambdamp\Phimp.
$$
Then, for any point $\xx$ in the  domain of propagation $\Omegaps$, the single-layer potential reads
\begin{equation}\label{eqEqInt:Lphimpt}
\Lop\rho(\xx) = \sum_{p=1}^M\sum_{m\in\Zb}\rhomp \Lop \Phimp(\xx) =\sum_{p=1}^M\sum_{m\in\Zb}\rhomp \frac{i\pi a_p}{2} J_m(ka_p)\Hm(kr_p(\xx)) \Phimpt(\xx),
\end{equation}
and the double-layer potential can be expressed as
\begin{equation}\label{eqEqInt:Mphimpt}
\Mop\lambda(\xx) = \sum_{p=1}^M\sum_{m\in\Zb}\lambdamp\Mop \Phimp(\xx) = -\sum_{p=1}^M\sum_{m\in\Zb} \lambdamp\frac{i\pi ka_p}{2} J_m'(ka_p)\Hm(kr_p(\xx)) \Phimpt(\xx).
\end{equation}
\end{prop}
Proposition \ref{propEqInt:LMphi} implies that, for any  $\xx$ in $\Omegaps$,
$$
 u(\xx) = \Lop\rho(\xx) + \Mop\lambda(\xx) = \sum_{p=1}^M\sum_{m\in\Zb}\frac{i\pi a_p}{2} \left[\rhomp  J_m(ka_p) + \lambdamp \Jmp(k\ap)\right]\Hm(kr_p(\xx)) \Phimpt(\xx).
$$

	\subsubsection{Inside the obstacles}

In the same way, the potentials can be computed inside the obstacles, which is useful for penetrable obstacles for instance. In that case however, only the contribution of the current obstacle is taken into account:
$$
\um(\xx) = \Lop_p\rho_p + \Mop_p\lambda_p,\qquad\forall\xx\in\Omega_p.
$$

\begin{prop}\label{propEqInt:Inside}
Let $\rho\in L^{2}(\Gamma)$ and $\mu\in H^{1/2}(\Gamma)$ be two densities admitting the following decompositions in the Fourier basis $\BF$ 
$$
\rho = \sum_{p=1}^{M}\sum_{m\in\Zb} \rhomp\Phimp \qquad\text{ and }\qquad \lambda = \sum_{p=1}^{M}\sum_{m\in\Zb} \lambdamp\Phimp.
$$
Then, for any point $\xx$ inside the obstacle $\Omegap$, the single-layer potential reads
\begin{equation}\label{eqEqInt:LphimptINT}
\Lop\rho(\xx) = \sum_{m\in\Zb}\rhomp \Lop \Phimp(\xx) =\sum_{p=1}^M\sum_{m\in\Zb}\rhomp \frac{i\pi a_p}{2} H_m^{(1)}(ka_p)J_m(kr_p(\xx)) \Phimpt(\xx),
\end{equation}
and the double-layer potential can be expressed as
\begin{equation}\label{eqEqInt:MphimptINT}
\Mop\lambda(\xx) = \sum_{p=1}^M\sum_{m\in\Zb}\lambdamp\Mop \Phimp(\xx) = -\sum_{p=1}^M\sum_{m\in\Zb} \lambdamp\frac{i\pi ka_p}{2} \Hmp(ka_p)J_m(kr_p(\xx)) \Phimpt(\xx).
\end{equation}
\end{prop}


%==============================================================================
	\subsection{Far-field and Radar Cross Section (RCS)}\label{secEqInt:SER}
%==============================================================================

For computing the far-field pattern, let us recall that the scattered field $u$ admits the following Helmholtz's integral representation:
$u = \Lop \rho + \Mop \lambda$,
where $\rho$ and $\lambda$ are two unknown densities. In the polar coordinates system $(r,\theta)$ and by using an asymptotic expansion
of  $u$ when $r\to +\infty$, the following relation holds \cite{ColKre83}
$$
\forall \theta\in [0,2\pi],\qquad u(r,\theta) = \frac{e^{ikr}}{r^{1/2}} \left[ a_{\Lop}(\theta) + a_{\Mop}(\theta) \right] + \GrandO{\frac{1}{r^{3/2}}},
$$
where $a_{\Lop}$ and $a_{\Mop}$ are the radiated far-fields  for the single- and double-layer potentials, respectively, defined for any angle
 $\theta$ of $[0,2\pi]$ by
$$
\begin{cases}
\dsp{a_{\Lop}(\theta) = \frac{1}{\sqrt{8k\pi}}e^{i\pi /4} \int_{\Gamma} e^{-ik \thetab \cdot \yy} \rho(\yy) \dd \Gamma(\yy),}\\[0.4cm]
\dsp{a_{\Mop}(\theta) = \frac{1}{\sqrt{8k\pi}}e^{i\pi /4} \int_{\Gamma} -\frac{ik}{\|\yy\|} \thetab \cdot \yy e^{-ik \thetab \cdot \yy} \lambda(\yy) \dd \Gamma(\yy),}
\end{cases}
$$
with $\thetab:=(\cos(\theta),\sin(\theta))$. In addition, the Radar Cross Section (RCS) is defined by
$$%\begin{equation}\label{eqEqInt:defSER}
\forall \theta\in[0,2\pi], \quad \textrm{RCS}(\theta) = 10\log_{10}\left(2\pi\Abs{\aLop(\theta) + \aMop(\theta)}^{2}\right) (\dB).
$$%\end{equation}
To optimize the far-fields computation, these relations can be written thanks to the inner product between two infinite vectors.
Indeed, let us introduce $\aabt_{\Lop} = ((\aabt_{\Lop})^{p})_{1\leqslant p \leqslant M}$ and $\aabt_{\Mop} = ((\aabt_{\Mop})^{p})_{1\leqslant p \leqslant M}$,
where $(\aabt_{\Lop})^{p}$ and $(\aabt_{\Mop})^{p}$ are given by:  $\forall p=1,\ldots,M$,
\begin{equation}
\label{eq:FFFourier}
\left\{\begin{array}{l}
\dsp{(\aabt_{\Lop})^{p} = \Big((\aabt_{\Lop})^{p}_{m}\Big)_{m\in\Zb}, \qquad (\aabt_{\Lop})^{p}_{m} = \frac{ie^{-i\pi/4}\sqrt{a_{p}}}{2\sqrt{k}} e^{-i\bp k\cos(\theta-\alphap)}\Jm(k\ap)e^{im(\theta-\pi/2)},}\\[0.3cm]
\dsp{(\aabt_{\Mop})^{p} = \Big((\aabt_{\Lop})^{p}_{m}\Big)_{m\in\Zb}, \qquad (\aabt_{\Lop})^{p}_{m} = \frac{ie^{-i\pi/4}\sqrt{ka_{p}}}{2} e^{-i\bp k\cos(\theta-\alphap)}\Jmp(k\ap)e^{im(\theta-\pi/2)}.}
\end{array}\right.
\end{equation}
Then, we obtain the following: $\dsp{a_{\Lop}(\theta) = (\aabt_{\Lop})^{T}\Rhot}$ and $\dsp{a_{\Mop}(\theta) = (\aabt_{\Mop})^{T}\lambdabt}$.


%----------------------------------------
	\section{Finite-dimensional approximations and numerical solutions proposed in $\mu$-diff}\label{sectionFiniteDimensional}
%----------------------------------------

We now have all the ingredients to numerically solve the four integral equations EFIE, MFIE, CFIE and BWIE,
for sound-soft obstacles. In fact, any integral equation for any boundary condition can be solved according to the previous developments.
In practice, the infinite Fourier systems need to be truncated to get a finite dimensional problem: we must pass from
a sum over  $m\in \mathbb{Z}$ to a finite number of Fourier modes that depends on $ka_{p}$, $p=1,...,M$.
Let us consider e.g. the EFIE, the extension to the other boundary integral operators being direct. The EFIE is given by
 equation (\ref{eqEqInt:MatLbt}): $\Lbt\Rhot = -\UUt$. To truncate each Fourier series associated with 
  $(\Phimp)_{m\in\Zb}$ for the obstacle $\Omegamp$, we only keep $2N_{p} +1$ modes in such a way that the indices $m$ of the truncated series
  satisfy: $\forall p=1,\ldots,M$, $-\Np\leqslant m \leqslant \Np$. The truncation parameter $N_{p}$ must be fixed large enough, with
  $N_{p} \geqslant k a_{p}$, for $p=1,...,M$. An example \cite{AntChnRam08,JACT} is: $N_{p} = k a_{p} +�C_{p}$, where
   $C_{p}$ weakly grows with $k a_{p}$. A numerical study of the parameter $\Np$ is proposed in \cite{AntChnRam08,JACT} where 
   the following formula leads to a stable and accurate computation
\begin{equation}\label{eq:Np}
 \Np = \left[ k a_p+ \left(\frac{1}{2\sqrt{2}} \ln (2\sqrt{2} \pi k a_p \varepsilon^{-1})\right)^{\frac{2}{3}} (k a_p)^{1/3} +1\right ],
\end{equation}
where $\varepsilon$ is a small parameter (related to the relative tolerance required in the iterative Krylov subspace solver used
for solving the truncated linear system (\ref{EFIEDimensionFinie}), see  \cite{AntChnRam08,JACT}).

The resulting  linear system writes
\begin{equation}\label{EFIEDimensionFinie}
\Lb\Rho = -\UU,
\end{equation}
where we introduced the block matrix $\Lb = (\Lbpq)_{1\leqslant p,q\leqslant M}$ and the vectors $\Rho = (\Rho^{p})_{1\leqslant p\leqslant M}$ and
 $\UU = (\UU^{p})_{1\leqslant p\leqslant M}$ defined by
\begin{equation}\label{eqEqInt:EFIEtronque}
\Lb = \left[
\begin{array}{c c c c}
\Lb^{1,1} & \Lb^{1,2} & \ldots & \Lb^{1,M} \\
\Lb^{2,1} & \Lb^{2,2} & \ldots & \Lb^{2,M} \\
\vdots & \vdots & \ddots & \vdots\\
\Lb^{M,1} & \Lb^{M,2} & \ldots & \Lb^{M,M}
\end{array}
\right],
\qquad \qquad
\Rho =
\left[
\begin{array}{c}
\Rho^{1} \\
\Rho^{2} \\
\vdots \\
\Rho^{M}
\end{array}
\right],\qquad \qquad
\UU =
\left[
\begin{array}{c}
\UU^{1} \\
\UU^{2} \\
\vdots \\
\UU^{M}
\end{array}
\right].
\end{equation}
For $p,q= 1,\ldots, M$,
 the  complex-valued matrix $\Lbpq$ is of size $(2\Np+1)\times(2N_{q}+1)$ and its coefficients $\Lbpqmn$ are: $\Lbpqmn = \Lbtpqmn$,
 for $m=-\Np,\ldots,\Np$, $ n = -\Nq,\ldots,\Nq$.
The complex-valued components of the vector $\Rho^{p} = (\Rho^{p}_{m})_{-\Np\leqslant m \leqslant\Np}$ of size $2\Np+1$ are
  the approximate Fourier coefficients  $\rhomp$ of $\rho$. For the sake of clarity, we keep on writing: $\Rho^{p}_{m} = \Rhot^{p}_{m} = \rhomp$,
  for all $ m =-\Np,\ldots,\Np$.
The complex-valued vector $\UU^{p} = (\UU^{p}_{m})_{-\Np\leqslant m \leqslant\Np}$ is composed of the  $2\Np+1$ Fourier
coefficients of the trace of the incident wave on $\Gamma$, i.e.  $\UU^{p}_{m} = \UUt^{p}_{m} = \PSGamma{\uincg}{\Phimp}$,
$\forall m =-\Np,\ldots,\Np$.
If $\Ntot = \sum_{p=1}^{M}(2\Np +1)$ denotes the total number of modes, the size of the complex-valued matrix
 $\Lb$ is then $\Ntot\times\Ntot$. More generally, all the boundary integral operators can be truncated according to this process.
 Concerning the notations, it is sufficient to formally omit the tilde symbol $\sim$ over the quantities involved in sections
  (\ref{integralinfinite}) to (\ref{secEqInt:SER}).

Since  the four finite-dimensional matrices $\Lb$, $\Mb$, $\Nb$ and $\Db$ that respectively correspond to the four
boundary integral operators $L$, $M$, $N$ and $D$ can be computed,  the linear systems that approximate the EFIE, MFIE, CFIE and
 BWIE can be stated. For example, the CFIE leads to (with $0\leqslant \alpha\leqslant 1$ and $\Im(\eta)\neq 0$)
\begin{equation}\label{eqEqInt:CFIEDapp}
\left[ \alpha\eta\Lb + (1-\alpha) \left(\frac{\Ib}{2} + \Nb\right)\right]\Rho = -\alpha\eta\UU - (1-\alpha)\dUU.
\end{equation}
Let us remark that the matrix obtained after discretization is always a linear combination of the four integral operators
 $\Lb$, $\Mb$, $\Nb$, $\Db$ and the identity matrix  $\Ib$. As a consequence, for a given integral equation,
 the resulting matrix is of size $\Ntot\times\Ntot$ and has the same block structure as e.g. 
  $\Lb$ (see equation  (\ref{eqEqInt:EFIEtronque})).
 The finite-dimensional linear system (\ref{EFIEDimensionFinie}) (or (\ref{eqEqInt:CFIEDapp})) is accurately solved in $\mu$-diff
by using the Matlab direct solver or a preconditioned Krylov subspace linear solver 
that uses fast matrix-vector products based on Fast Fourier Transforms (FFTs), the choice of the linear algebra strategy
(direct vs. iterative) depending on the configuration with respect to $ka_{p}$ and  
$M$. The use of FFTs is made possible since the off-diagonal blocks of the integral operators can be written as the products
of diagonal and Toeplitz matrices \cite{AntChnRam08,JACT} (see e.g. the matrices $\Sbtpqmn$ in section \ref{integralinfinite}). In addition,
low memory is only necessary when $ka_{p}$ is large enough since  the storage of the Toeplitz
matrices can be optimized. This resulting storage technique is called \textit{sparse} representation in $\mu$-diff,
 in contrast with the \textit{dense} (full) storage of the complex-valued matrices.
 Let us assume that $a_p \approx a$, for $1 \leqslant p \leqslant M$.
In terms of storage, the dense version of a matrix requires to store about $4M^{2}[ka]^{2}$ coefficients (assuming that $N_{p}$ are fixed
by formula (\ref{eq:Np}), and $[r]$ denotes the integer part of a real number $r$) while the sparse storage needs about
$4M^{2}[ka]$ complex-valued coefficients. In terms of computational time for solving the linear system,
the direct (multithreaded) gaussian solver included in Matlab leads to a cost that scales with $\mathcal{O}(M^{3}(ka)^{3})$. For the preconditioned iterative Krylov subspace
methods (i.e. restarted GMRES)), the global cost is $\mathcal{O}(M^2 ka \log_{2}(ka))$, the converge rate depending on
the physical situation and robustness of the preconditioner. From these remarks, we deduce that an iterative method is an efficient  and cheap 
alternative to a direct solver for large wavenumbers $ka$, but also for large $M$.
 We refer to \cite{AntChnRam08,JACT} for a thorough computational study of the various numerical strategies.
A few examples in $\mu$-diff are provided with the toolbox.
Finally, the post-processing formulas (near- and far-fields quantities) clearly inherits of the truncation procedure (see sections \ref{secEqInt:evaluation} and \ref{secEqInt:SER}).


\chapter{\mudiff Toolbox}\minitoc
\label{chap:code}
\section{Generalities}
The authors try to keep the notations between the mathematical background and the \matlab toolbox. In particular, lots of arguments are the same in the different functions and they are here presented once for all. A manual of every function is moreover provided by simply typing in the \matlab command window 
\begin{verbatim}
help name_of_the_function
\end{verbatim}
In all what follows, and if nothing is specified, then the following arguments refer to the below ones. The indices $p$ and $q$ vary from $1$ to $\Nscat$ where $\Nscat$ is the number of obstacles.

\paragraph{Geometry:}

\begin{center}
%\rowcolors{1}{white}{gray}
\begin{tabular}{|c |c | p{10cm}|}
\hline Name & Size or Type & Content\\[0.2cm]\hline\hline
\code{N\_scat} & $[1\times 1]$ & Number of obstacles $\Nscat$.\\\hline
\code{O} & $[2\times \Nscat]$ & Matrix of the centers of the disks such that \code{O(1,p)} is the x-coordinate of the $p^{th}$ obstacle.\\\hline
\code{Op} & $[2\times 1]$ & Coordinate of the $p^{th}$ scatterer.\\\hline
\code{Oq} & $[2\times 1]$ & Coordinate of the $q^{th}$ scatterer.\\\hline
\code{a} & $[1\times \Nscat]$ & Vector of the radii of the disks such that \code{a(p)} is the radius of the $p^{th}$ scatterer.\\\hline
\code{ap} & $[1\times 1]$ & Radius of the $p^{th}$ scatterer.\\\hline
\code{aq} & $[1\times 1]$ & Radius of the $q^{th}$ scatterer.\\\hline
\end{tabular}
\end{center}

\paragraph{Parameters (wavenumbers, incident waves, fourier series,\ldots):}
\begin{center}
\begin{tabular}{|c |c | p{10cm}|}
\hline Name & Size or Type & Content\\[0.2cm]\hline\hline
\code{beta\_inc} & $[1\times 1]$ & Angle of direction of a plane wave $e^{ik (\cos(\beta)x + \sin(\beta)y)}$.\\\hline
\code{XS} & $[2\times 1]$ & Center $(x_s,y_s)$ of a point source: $x_s=$\code{XS(1)} and $y_s=$\code{XS(2)}. A point source wave is then given by $i/4\Hz(k\|\xx-\xx_s\|)$, with $\xx=(x,y)$ and $\xx_s=(x_s,y_s)$ and $\Hz$ the zeroth order Hankel function of first kind.\\\hline
\code{k} & $[1\times 1]$ & Wavenumber $k$ in the vacuum.\\\hline
\code{k\_int} & $[1\times \Nscat]$ & Wavenumber in the obstacles: $\kintp=$\code{k\_int(p)}. If \code{k\_int} is a scalar then $\kintp$ = \code{k\_int} for all $p$.\\\hline
\code{M\_modes} & $[1\times \Nscat]$ & Vector of index of truncation of the Fourier series, \ie \code{M\_modes(p)}=$\Np$\\\hline
\code{Np} & $[1\times 1]$ & Corresponds to $\Np$, the truncation index in the Fourier serie.\\\hline
\code{Nq} & $[1\times 1]$ & Corresponds to $\Nq$, the truncation index in the Fourier serie.\\\hline
\end{tabular}
\end{center}

\paragraph{Incident wave:}
\begin{center}
\begin{tabular}{|c |c | p{10cm}|}
\hline Name & Size or Type & Content\\[0.2cm]\hline\hline
\code{PlaneWave} & $[1\times 1]$ & Wavenumber in the vacuum.\\\hline
\end{tabular}
\end{center}

\paragraph{Integral operators:} they are indexed and named by the following table:
\begin{center}
\begin{tabular}{|c |c | c | p{9cm}|}
\hline Index & Letter & Operator & \mudiff abreviation \\\hline\hline
0 & - & - & Null operator \\
1 & $I$ & \code{Identity} & Identity\\
2 & $L$ & \code{SingleLayer} & Trace of the single-layer\\
3 & $M$ & \code{DoubleLayer} & Trace of the double-layer\\
4 & $N$  & \code{DnSingleLayer}& Normal derivative of the single-Layer\\
5 & $D$  & \code{DnDoubleLayer}&Normal derivative of the double-Layer\\ 
6 & $\hat{L}$  & \code{PrecondDirichlet}& Single-scattering preconditioned trace of the single-layer operator\\
7 & $\hat{D}$  & \code{PrecondNeumann}& Single-scattering preconditioned normal derivative trace of the double-layer operator\\ \hline
\end{tabular}
\end{center}


\section{Pre-Processing}

The pre-processing functions are located in the \code{PreProcessing} folder, which is divided in two parts: the construction of the obstacles in \code{Geometry}, and the computation of the right-hand side, \ie the incident waves on the obstacles, in \code{IncidentWave}.

\subsection{Creating Obstacles}

\subsubsection{Manual placement}
The disks can be created manually by simply creating the two variables \code{O} and \code{a} containing respectively the coordinates of the disks and their radii. For example, for three obstacles placed on $(-1,2)$, $(5,5)$ and $(-15,10)$ with radii $0.1$, $0.5$ and $10$:
\begin{verbatim}
O = [-1, 5, 2 ; -15, 5, 10];
a = [0.1, 0.5, 10];
\end{verbatim}

\subsubsection{Periodic placement}

Two build-in functions are available with the toolbox to create disks periodically, with a rectangular or a triangular lattice, as shown on figure \ref{??}. The two functions are called as follows, for the rectangular lattice:
\begin{verbatim}
O = RectangularLattice(bx, by, Nx, Ny);
\end{verbatim}
and for the triangular lattice:
\begin{verbatim}
O = TriangularLattice(bx,by,Nx,Ny);
\end{verbatim}
where :
\begin{itemize}
\item \code{bx} and \code{by} are the distance separating two centers in the $x$ coordinate and \code{by} the distance between two rows in the $y$-direction. The two other parameter 
\item \code{Nx}: number of disks in a row
\item \code{Ny}: number of rows
\end{itemize}
For both functions, the vector of radii must be built separately. If the disks are unitary then the following command can be used
\begin{verbatim}
a = ones(size(O,2));
\end{verbatim}


\subsubsection{Random placement}

Finally, to place randomly obstacles in a box $[\code{xmin}, \code{xmax}]\times[\code{ymin},\code{ymax}]$ with also a random radius, a function \code{CreateRandomDisks} is very helpful. In its simplest version, the function is called as:
\begin{verbatim}
[O, a] = CreateRandomDisks(xmin, xmax, ymin, ymax, N_scat);
\end{verbatim}
The function will then create \code{N\_scat} disk with unit radius in the desired box. This is however probably too simple and the function can be called in a more complex form:
\begin{verbatim}
[O, a] = CreateRandomDisks(xmin, xmax, ymin, ymax, N_scat, 
           amin, amax, dmim, dmax, O_avoid, a_avoid, dmin_avoid, dmax_avoid);
\end{verbatim}
where
\begin{center}
\begin{tabular}{|c |c|c | p{10cm}|}
\hline Variable & Type & Default & Action\\\hline
\code{amin} & scalar  & 1 & Minimal (random) radius of the obstacles allowed \\\hline
\code{amax} & scalar  & 1 & Maximal (random) radius of the obstacles  allowed\\\hline
\code{dmin} & scalar & \code{realmin} & Minimal distance allowed between two obstacle (not between the centers!). Setting $\leq 0$  value will set \code{dmin} to \code{realmin} (\ie ignore it)\\\hline
\code{dmax} & scalar & \code{realmax} & Maximal distance allowed between two obstacle (not between the centers!). The maximal distance is quickly reached! Setting $\leq 0$  value will set \code{dmax} to \code{realmax} (\ie ignore it).\\\hline
\code{O\_avoid} & \code{[2 x N]} & \code{[]} & Center of \code{N} hole(s) where the obstacles must not overlap. Usefull for example for point source location.\\\hline
\code{a\_avoid} & \code{[1 x N]} & \code{[]} & Radii of the \code{N} holes\\\hline
\code{dmin\_avoid} & \code{[1 x N]} & \code{[]} & Minimal distance between an obstacle and a hole\\\hline
\end{tabular}
\end{center}
These optional arguments are not mandatory in the function call. For example:
\begin{verbatim}
[O, a] = CreateRandomDisks(xmin, xmax, ymin, ymax, N_scat, amin, amax);
\end{verbatim}
will create random disks with random radii without taking into account the distances between the disks (except obviously the overlapping).

For example, building 7 obstacles in the box $[-10,10]\times[-10,10]$ with radii between $0.1$ and $0.5$. The disks must be separated at minimum by a distance of $0.1$ and without maximum value. The command is then:
\begin{verbatim}
[O, a] = CreateRandomDisks(-10, 10, -10, 10, 7, 0.1, 0.5, 0.1, -1);
\end{verbatim}
Now imagine that a point source is located on $(2,2)$ and that the obstacles must be separated from the source from at least $0.3$, then the ``\code{*\_avoid}'' arguments can be used and command can be
\begin{verbatim}
[O, a] = CreateRandomDisks(-10, 10, -10, 10, 7, 0.1, 0.5, 0.1, -1, [2;2], 0.3);
\end{verbatim}
the disk centered on $(2,2)$ with radius $0.3$ will then be avoided. A second option is to set \code{a\_void} to zero and set the minimal distance \code{dmin\_avoid} to $0.3$:
\begin{verbatim}
[O, a] = CreateRandomDisks(-10, 10, -10, 10, 7, 0.1, 0.5, 0.1, -1, [2;2], 0, 0.3);
\end{verbatim}


\begin{remark}
To verify if a disk is well placed, \code{CreateRandomDisks} calls \code{CheckPlacement} function, which can also be useful for a user placing obstacles.
\end{remark}

\subsubsection{Removing disks}

The function \code{RemoveDisk} aims to remove some disks of the geometrical configuration, either disk by disk, by row or by column or by radius. Here is its syntax
\begin{verbatim}
[O,a] = RemoveDisk(O_old, a_old, ...);
\end{verbatim}
where \code{O\_old} and \code{a\_old} are the centers and radii of the current geometry. Without optional argument, the function will return \code{[O\_old, a\_old]} and the available arguments are:
\begin{itemize}
\item \code{[O,a] = RemoveDisk(..., 'X', [X1, X2, ..., XN]);}\\
Remove all the points with X abscissa X1, X2, ..., or XN
\item \code{[O,a] = RemoveDisk(...,  'Y', [Y1, Y2, ..., YN]);}\\
Remove all the points with Y ordinate Y1, Y2, ..., or YN
\item \code{[O,a] = RemoveDisk(..., 'XY', [[X1;Y1], [X2;Y2], ..., [XN;YN]]);}\\
Remove all the points [X1;Y1], [X2;Y2], ..., and [XN;YN]
\item \code{[O,a] = RemoveDisk(..., 'Radius', [a1, a2, ..., aN]);}\\
Remove all the disk with radius a1, a2, ..., or aN
\item \code{[O,a] = RemoveDisk(..., 'Verbosity', VERBOSITY);}\\
set VERBOSITY to 0 to avoid display message, to 1 to only show results, and to $>1$ to see everything (default).
\end{itemize}

For example, to remove every obstacle on the row of $x-$abscissa $1$ and $y-$ordinate $2.5$:\\
\code{  [O,a] = RemoveDisk(O\_old, a\_old, 'X', 1, 'Y', 2.5);}\\
Or, to remove the obstacles centered on $(2,5)$ and (3,4):\\
\code{  [O,a] = RemoveDisk(O\_old, a\_old, 'XY', [2,3;5,4]);}\\

\subsection{Incident waves}

Two different incident waves, plane wave and point source wave, are available in the \mudiff toolbox, but it should be highlighted that the user can build his/her own incident wave. A right-hand side $b$ is decomposed by blocks, each of these representing one obstacle: $b= (b_p)_{p=1,\ldots,\Nscat}$. As a different condition can be applied on an obstacle or a different integral equation can be considered on it, each block can be specified separately. Depending on the type of desired right-hand side, the block function \code{BlockIncidentWave} will compute the vector $b_p$. The frontal function \code{IncidentWave}, which compute the whole vector $b$, consists on calling  \code{BlockIncidentWave} for each obstacle and assembling the vector. Interfaced functions are available to compute directly some incident wave, such as \code{PlaneWave}, \code{PointSource} or \code{DnPlaneWave} and \code{DnPointSource}. They call \code{IncidentWave} with the right argument. There is no need here to described these functions, however their help is helpful and contains the mathematical description of the incident wave.

\subsubsection{\code{BlockIncidentWave}}







\section{Integral operators}

Two different type of storage are provided with the \mudiff toolbox: dense and sparse. The dense version store the whole matrix in memory whereas the sparse version uses the special structure of the matrix of an integral operator to store it. The sparse storage in \mudiff and this user guide has nothing to deal with the sparse storage provided in \matlab such as \texttt{sparse} function. The dense storage is easier to use and works pretty well for small scale problems. It also presents the advantage of providing the whole matrix of the integral operator, which can be useful for spectrum analysis for example. On the other hand, for a large number of circular obstacles and/or for large frequency, the memory storage becomes too important and the sparse version must be used. One should be however careful: the sparse matrix-vector product, based on the cross-correlation (\texttt{xcorr} \matlab function), is very sensitive to the number of modes chosen in the truncation of the Fourier series. Indeed, if too many modes are kept, the matrix-vector product show to be unstable. The formula (\ref{??}) seems to provide stability.

For both dense and sparse case, let $\Ab$ be a generic matrix representing the matrix of one of the four boundary integral operator, $\Lb,\Mb,\Nb$ and $\Db$. As highlighted in previous chapter, $\Ab$ has the following structure, for $p,q=1,\ldots,\Nscat$ and $p\neq q$:
\begin{itemize}
\item $\Abpp$ is diagonal.
\item $\Abpq$ is full and can be divided as $\Abpq = \AbpqL\Tbpq\AbpqR$ where $\AbpqL$ and $\AbpqR$ are diagonal and called respectively the left and right part, and $\Tbpq = (\Tbpqmn))$, with $\Tbpqmn = i\pi e^{??}H_0^{(1)}(k\bpq)$, is a Toeplitz matrix.
\end{itemize}
In the sparse version, diagonal submatrices $\Abpp,\AbpqL$ and $\AbpqR$  are stored as a vector of size respectively $2\Np+1$, $2\Np+1$ and $2\Nq+1$, and the Toeplitz matrices $\Tbpq$ are stored as vectors of size $2\Np+2\Nq-1$.

This section is naturally divided in two part, the first being devoted to the dense storage and the second to the sparse version.

\subsection{Available integral operators}

The integral operators are numbered as follows
\begin{enumerate}
\item Null operator
\end{enumerate}


\subsection{Dense storage}
\subsubsection{Block structures}

A matrix $\Ab$ is created by blocks using \texttt{BlockIntegralOperator} function, which has the following syntax
\begin{verbatim}
	Apq = BlockIntegralOperator(Op, ap, Np, Oq, aq, Nq, k, TypeOfOperator)
\end{verbatim}
where:
\begin{itemize}
\item \texttt{Op}(resp. \texttt{Oq}): \texttt{2$\times$1}, vector containing the centers of the disk such that \texttt{Op(1,p)}$=x_{\OOp}$ and \texttt{Op(2,p)}$=y_{\OOp}$ (resp. \texttt{Oq(1)}$=x_{\OOq}$ and \texttt{Oq(2)}$=y_{\OOq}$)
\item \texttt{ap}(resp. \texttt{aq}): \texttt{1$\times$1}, Radius of the disks $\OOp$ (resp. $\OOq$)
\item \texttt{Np} (resp. \texttt{Nq}): \texttt{1$\times$1}, Index of truncation $\Np$ (resp. $\Nq$)
\item \texttt{k}: \texttt{1$\times$1}, Wavenumber in the vacuum
\item \texttt{TypeOfOperator}: \texttt{Variable size}, Type of integral operator to be computed.
\end{itemize}

\subsubsection{Full matrix}

\subsection{Sparse storage}
\subsubsection{Block structures}
\subsubsection{Full matrix}

\section{Post-Processing}
\subsubsection{Near field}
\subsubsection{Far field and Radar Cross Section (RCS)}
\subsubsection{Plot and display}



\chapter{Full examples}\minitoc
\label{chap:examples}
The aim of this chapter is to provide some examples of multiple scattering problems solved by  the \mudiff toolbox.
The impenetrable case with a Dirichlet, a Neumann or a mixed of both boundary conditions  set on the boundaries of the obstacles are
fully treated.


\section{The Dirichlet boundary-value problem}

Let us consider the scattering problem by a collection of sound-soft obstacles
$$
\left\{\begin{array}{r c l l}
(\Delta +k^2)u & = & 0, & \text{ in }\Omegaps,\\
u & = & -\uinc, & \text{ on }\Gamma,\\
\multicolumn{4}{l}{\qquad \qquad u \text{ outgoing},}
\end{array}\right.
$$
with  $\Omegam = \bigcup_{p=1}^{M}\Omegamp$.
We propose to solve this problem through various integral equations: the EFIE (\ref{eq:EFIE}), the MFIE (\ref{eq:MFIE}), the 
CFIE (\ref{eq:CFIE}) and the single-scattering preconditioned integral equations (\ref{eqEqINt:LAsglLA}). We show how
 to use both the full and  sparse storages of the matrices.
Before starting, we recommend to use the  single-scattering preconditioned integral equation as presented in \S\ref{secEx:PrecondD}. 
Indeed, the resulting system is well-posed and is well-conditioned leading to an efficient solution by a Krylov subspace iterative solver.

\subsection{Pre-processing}

Let us first consider a collection of three sound-soft unit circular cylinders. The  wavenumber is 
$k=2\pi$ and the direction of  incidence of the
 wave is $\beta = 0$ degree. The resulting  \mudiff pre-processing code for setting  these parameters is then
\begin{lstlisting}
%% Pre-processing
% Three unit disks 
O = [-5, 0, 5; -2, 0, 2];
a = [1, 1, 1];
%Set the parameters...
k = 1; %wavenumber
beta_inc = 0; %incident angle
%Fourier series truncation parameter
M_modes = FourierTruncation(a, k, 'Min', 1);
\end{lstlisting}

For each integral equation, we now present the assembly process, the computation of the solution and finally the post-processing of the computed wave
fields. 
The common pre-processing part is the one described above. All the functionalities presented here are also available in  the
 file \texttt{BenchmarkDirichlet.m} which is located in the  \folder{Examples/Benchmark} folder.
 To launch the computations, it is sufficient to type the following command
\begin{lstlisting}
BenchmarkDirichlet;
\end{lstlisting}
in the Matlab window once $\mu$-diff has been installed.

\subsection{The case of the EFIE}


This integral formulation reads as
$$
\left\{\begin{array}{r c l}
u &=& \Lop\rho,\\
L\rho &=& -\uinc|_{\Gamma},
\end{array}\right.
$$
where the first line is the integral equation representation of the exterior wavefield $u$ and the second one is  the surface 
integral equation to solve. 

\subsubsection{Dense storage}

In \mudiff, the surface single-layer operator $L$ and the incident plane wave field $\uinc|_{\Gamma}$ are  predefined quantities.
If the full storage of the integral equation is used, then the direct solution of the resulting linear system can be
obtained by the standard backslash Matlab operator $\backslash$ 
\begin{lstlisting}
%Right-hand side
Uinc = PlaneWave(O, a, M_modes, k, beta_inc);
%% Assembling
%Matrix of the system (the two following lines are the same)
L = SingleLayer(O, a, M_modes, k);
%% Solving (here, direct)
rho = L \ Uinc;
\end{lstlisting}
\medskip

Once the surface wavefield has been computed, the RCS can be calculated by the following $\mu$-diff  commands
\begin{lstlisting}
%% Post-processing
%Scattering angles 
theta_RCS = 0:360;
theta_RCS_rad = theta_RCS*2*pi/360;
%Radar Cross Section for the single-layer representation (<-> [1,0])
myRCS = RCS(O, a, M_modes, k, theta_RCS_rad, rho, [1,0]);
plot(theta_RCS, myRCS, 'k');
\end{lstlisting}

\subsubsection{Sparse storage}

For the sparse storage version, only the assembly process of the single-layer matrix and the system solution
need to be modified as follows
\begin{lstlisting}
%Matrix of the system (the two following lines are the same)
SpL = SpSingleLayer(O, a, M_modes, k);
%% Solving (here, direct)
rho = gmres(@(X)SpMatVec(X, M_modes, SpL), Uinc);
\end{lstlisting}
\medskip

\subsection{The case of the MFIE}


The resolution of the scattering problem by the MFIE (\ref{eq:MFIE}) leads to the integral equation representations
$$
\left\{\begin{array}{r c l}
u &=& \Lop\rho,\\
\dsp \left(\frac{I}{2}+N\right)\rho &=& -\dn\uinc|_{\Gamma}.
\end{array}\right.
$$

\subsubsection{Dense storage}

The MFIE operator $$\left(\frac{I}{2}+N\right)$$ can be computed  thanks to the frontal function \IntegralOperator with two
 arguments: the type of the operators (for the identity operator and the double-layer potential operator $N$, see
  Table \ref{table:IntOp}) and their associated weights ($0.5$ and $1$). 
\begin{lstlisting}
%Right hand side
DnUinc = DnPlaneWave(O, a, M_modes, k, beta_inc);
%% Assembling
%Matrix of the system (the two following lines are the same)
A_MFIE = IntegralOperator(O, a, M_modes, k, [1, 4], [0.5, 1]);
%% Solving (here, direct)
rho = A_MFIE \ DnUinc;
\end{lstlisting}
\medskip

The post-processing part is exactly the same as for the EFIE since the surface equation is based on the 
volume single-layer integral representation.

\subsubsection{Sparse storage}

The sparse storage version is almost the same as for the dense storage except for  assembling  the matrix and solving the linear system. Indeed, the matrices
 $I$ and $N$ cannot be computed by the same function since the sparse  function representations \SpIntegralOperator and
  \IntegralOperator cannot be  summed together. It is however possible to add the identity to a ''sparse operator'' thanks to \SpAddIdentity
\begin{lstlisting}
SpN = SpDnSingleLayer(O, a, M_modes, k);
%Add I/2 to N:
SpA_MFIE = SpAddIdentity(SpN, 0.5, M_modes)
rho = gmres(@(X)SpMatVec(X, M_modes, SpA_MFIE), DnUinc);
\end{lstlisting}
\medskip

\subsection{The case of the CFIE}

Let us now consider  the well-posed and well-conditioned CFIE (see also Eq. (\ref{eq:CFIE}))
$$
\left\{\begin{array}{r c l}
u &=& \Lop\rho,\\
\dsp \left[\alpha\eta L  + (1-\alpha)\left(\frac{I}{2}+N\right)\right]\rho &=& -\alpha\eta\uinc|_{\Gamma} - (1-\alpha) \dn\uinc|_{\Gamma}.
\end{array}\right.
$$
Here, we fix the parameters  to $\alpha=0.5$ and $\eta = i/k$. 

\subsubsection{Dense storage}

The operator $$(1-\alpha)\left(\frac{I}{2}+N\right) + \alpha\eta$$ is computed in \mudiff by
using the \IntegralOperator function,  the post-processing remaining  unchanged,
\begin{lstlisting}
%CFIE
alpha = 0.5;
eta = i/k;
%Right-hand side
Uinc = PlaneWave(O, a, M_modes, k, beta_inc);
DnUinc = DnPlaneWave(O, a, M_modes, k, beta_inc);
BCFIE = alpha*eta*Uinc + (1-alpha)*DnUinc;
%% Assembling
%Matrix of the system (the two following lines are the same)
ACFIE = IntegralOperator(O, a, M_modes, k, [2, 1, 4], [alpha*eta, 0.5*(1-alpha), 1-alpha]);
%% Solving (here, direct)
rho = ACFIE \ BCFIE;
\end{lstlisting}
\medskip

\subsubsection{Sparse storage}

The sparse storage version changes compared to the dense one: the operators $I/2 + N$ and $L$ are computed separately and merged during the matrix-vector products. This is done in the \SpMatVec function
\begin{lstlisting}
SpL = SpSingleLayer(O, a, M_modes, k);
SpN = SpDnSingleLayer(O, a, M_modes, k);
SpA_MFIE = SpAddIdentity(SpN, 0.5, M_modes)
%% Solving and combining operators:
rho = gmres(@(X)SpMatVec(X,M_modes,{SpL, SpA_MFIE}, [alpha*eta, 1-alpha]), B_CFIE);
\end{lstlisting}
\medskip

\subsection{The case of the single-scattering preconditioned integral equation}
\label{secEx:PrecondD}
We strongly recommend
 to use the single-scattering preconditioned version of the EFIE, which is rigorously the same as the MFIE and CFIE and, up to an invertible operator, 
 to any other boundary integral equation (see Proposition \ref{prop:SingleScat}). The EFIE version is available in \mudiff and is represented as
$$
\begin{cases}
u = \Lop\rho,\\
\Lsgl^{-1} L \rho = -\Lsgl^{-1}\uinc.
\end{cases}
$$

\subsubsection{Dense storage}

In \mudiff, the quantity $-\Lsgl^{-1}\uinc|_{\Gamma}$ is provided by \PlaneWavePrecond whereas $\Lsgl^{-1} L$ is obtained with \PrecondDirichlet. The syntax for the dense version is then the following
\begin{lstlisting}
[...]
%Right-hand side
UincPrecond = PlaneWavePrecond(O, a, M_modes, k, beta_inc);
%Matrix of the system (the two following lines are the same)
APrecond = PrecondDirichlet(O, a, M_modes, k);
%Solving (here, directly)
rho = APrecond \ UincPrecond;
[...]
\end{lstlisting}
\medskip

\subsubsection{Sparse storage}

The sparse storage is here almost the same as for the dense version thanks to \SpPrecondDirichlet
\begin{lstlisting}
SpPrecond = SpPrecondDirichlet(O, a, M_modes, k);
%% Solving and combining operators:
rho = gmres(@(X)SpMatVec(X,M_modes, SpPrecond), UincPrecond);
\end{lstlisting}
\medskip

\section{The Neumann boundary-value problem}

Let us now consider the sound-hard scattering problem
$$
\left\{\begin{array}{r c l l}
(\Delta +k^2)u & = & 0, & \text{ in }\Omegaps,\\
\dn u & = & -\dn \uinc, & \text{ on }\Gamma,\\
\multicolumn{4}{l}{\qquad \qquad u \text{ outgoing}.}
\end{array}\right.
$$

An efficient solution to this problem is given for example by a preconditioned integral equation for sound-hard obstacles.
Here, we only present this solution but the extension to other kinds of integral equations is direct.
 The $\mu$-diff script is close to the one developed for the Dirichlet problem, only the two following functions must
  be modified: \PrecondDirichlet is replaced by \PrecondNeumann and the right-hand side \PlaneWavePrecond is now given by
   \DnPlaneWavePrecond. For the Neumann problem, the preconditioned boundary integral equation is based on the double-layer representation
$$
\begin{cases}
u = \Mop\lambda,\\
\widehat{D}^{-1} D \lambda = -\widehat{D}^{-1}\dn\uinc.
\end{cases}
$$
\begin{lstlisting}
% Three unit disks 
O = [-5, 0, 5; -2, 0, 2];
a = [1, 1, 1];
%Set the parameters...
k = 1; %wavenumber
beta_inc = 0; %incident angle
%Fourier series truncation parameter
M_modes = FourierTruncation(a, k, 'Min', 1);
%Right-hand side
DnUincPrecond = DnPlaneWavePrecond(O, a, M_modes, k, beta_inc);
%Matrix of the system (the two following lines are the same)
APrecond = PrecondNeumann(O, a, M_modes, k);
%Solving (here, direct)
lambda = APrecond \ DnUincPrecond;
\end{lstlisting}
\medskip

The post-processing is based on the double-layer potential (compared to Dirichlet, the modification is realized in the \RCS function, the
 last argument \code{[1,0]} is then replaced by  \code{[0,1]})
\begin{lstlisting}
%Scattering angles
theta_RCS = 0:360;
theta_RCS_rad = theta_RCS*2*pi/360;
%Radar Cross Section associated with the double-layer potential
myRCS = RCS(O, a, M_modes, k, theta_RCS_rad, lambda, [0,1]);
plot(theta_RCS, myRCS, 'k');
\end{lstlisting}
\medskip

\section{Mixing Dirichlet and Neumann boundary conditions}

Let us consider the following situation where we mix Dirichlet and Neumann boundary conditions. The scatterer is composed of
 $M_D$ sound-soft and  $M_N = M-M_D$ sound-hard obstacles, leading to the scattering problem
$$
\left\{\begin{array}{r c l l}
(\Delta +k^2)u & = & 0, & \text{ in }\Omegaps,\\
u & = & -\uinc, & \text{ on }\Gamma_p,\quad p=1,\ldots, M_D,\\
\dn u & = & -\dn \uinc, & \text{ on }\Gamma_p,\quad p=M_D+1,\ldots, M,\\
\multicolumn{4}{l}{\qquad \qquad u \text{ outgoing}.}
\end{array}\right.
$$
For this problem, the  preconditioned integral is not directly available. We can apply a Combined Field Integral Equation for the mixed problem
 (see equation (\ref{eq:CFIEMixte2})) and written as $$(\frac{I}{2} +A)\varphi = b,$$ where
  the matrix $A$ is
  given by  (\ref{eq:CFIEMixteApq}). We have
$$
A(p,q) = 
\begin{cases}
(1-\alpha) N^{p,q} + \alpha\eta L^{p,q}, & \text{ if } q \leq M_D,\\
(1-\alpha) D^{p,q} + \alpha\eta M^{p,q}, & \text{ if } q > M_D.\\
\end{cases}
$$
The  matrix is particularly easy to build  with \mudiff thanks to the frontal function \IntegralOperator. To this end, two
 three-dimensional arrays, \code{Assembling} and \code{Weight}, are built such that
$$
\code{Assembling}(:, p,q) = 
\begin{cases}
[4,2], & \text{ if } q \leq M_D,\\
[5,3], & \text{ if } q > M_D,
\end{cases}\qquad\text{and}\qquad
\code{Weight}(:, p,q) = [(1-\alpha), \alpha\eta].
$$
The indices in \code{Assembling} corresponds to the indices of the boundary integral operators ($2=L^{p,q}$, $3=M^{p,q}$, $4=N^{p,q}$, $5=D^{p,q}$). The assembling process is realized by \IntegralOperator.
\begin{lstlisting}
% Two Dirichlet obstacles (unit diks)
OD = [-5, -5; -5, 5];
aD = [1, 1];
N_scatD = length(aD);
% Two Neumann obstacles (unit diks)
ON = [5, 5; -5, 5];
aN = [1, 1];
N_scatN = length(aN);
%All obstacles
O = [OD, ON];
a = [aD, aN];
N_scat = N_scatD + N_scatN;
%Set the parameters...
k = 1; %wavenumber
beta_inc = 0; %incident angle
%Fourier series truncation parameter (Dirichlet, Neumann, All)
M_modesD = FourierTruncation(aD, k, 'Min', 1);
M_modesN = FourierTruncation(aN, k, 'Min', 1);
M_modes = [M_modesD, M_modesN];
%Right-hand side
Uinc = DnPlaneWave(O, a, M_modes, k, beta_inc);
DnUinc = DnPlaneWave(O, a, M_modes, k, beta_inc);
B = alpha*eta*Uinc + (1-alpha)*DnUinc;
%% Assembling
Assembling = zeros(2, N_scat, N_scat);
Weight = zeros(2, N_scat, N_scat);
for p=1:N_scatD
	for q=1:N_scat
		Assembling(:,p,q) = [4;2];
		Weight(:,p,q) = [1-alpha; alpha*eta];
	end
end
for p=N_scatD+1:N_scat
	for q=1:N_scat
		Assembling(:,p,q) = [5;3];
		Weight(:,p,q) = [1-alpha; alpha*eta];
	end
end
%Frontal function
A = IntegralOperator(O, a, M_modes, k, Assembling, Weight);
%Solving (here, direct)
density = A \ B;
\end{lstlisting}
\medskip

The post-processing is then done by specifying to \mudiff how the density must be used: for the first $M_D$ obstacles, a single-layer potential is used,
while for the others, a double-layer potential is required. The \RCS function can simply do that. It just needs an array \code{TypeOfOp} of size
 $\code{N\_scat}\times 2$ such that
$$
\code{TypeOfOp}(p,:) = \begin{cases}
[1,0], &\text{ if } p \leq M_D,\\
[0,1], &\text{ if } p > M_D.
\end{cases}
$$
In a \mudiff script, this  means ''Apply the single-layer potential (multiplied by $1$) for the first $M_D$ part of the density and a double-layer potential
 (multiplied by $1$) for the others''.
 \newpage
\begin{lstlisting}
%Preparing TypeOfOp
TypeOfOp = zeros(N_scat, 2);
for p =1:N_scat
	if(p <= N_scatD)
		TypeOfOp(p,1) = 1;
	else
		TypeOfOp(p,2) = 1;
	end
end
%Scattering angles
theta_RCS = 0:360;
theta_RCS_rad = theta_RCS*2*pi/360;
%Radar Cross Section computation
myRCS = RCS(O, a, M_modes, k, theta_RCS_rad, density, TypeOfOp);
plot(theta_RCS, myRCS, 'k');
\end{lstlisting}





%\chapter*{Conclusion}
%\mtcaddchapter[Conclusion]                          % solution pour minitoc
\markboth{\uppercase{Conclusion}}{\uppercase{Conclusion}} 

\ldots

\appendix

\chapter{List of \mudiff functions (alphabetic order)}\minitoc
\funByNameEven{BenchmarkCalderon.m}{Examples/Benchmark}{Example of solution of penetrable scattering using Calderon projectors}
\funByNameUneven{BenchmarkDirichlet.m}{Examples/Benchmark}{Example of solution of sound soft scattering using different integral equations}
\funByNameEven{BenchmarkNeumann.m}{Examples/Benchmark}{Example of solution of sound hard scattering using different integral equations}
\funByNameUneven{BenchmarkPenetrable.m}{Examples/Benchmark}{Example of solution of penetrable scattering using single layer potential}
\funByNameEven{BlockCalderonProjector.m}{IntOperators/Dense/Interface/Block}{Calderon projector block}
\funByNameUneven{BlockDnDoubleLayer.m}{IntOperators/Dense/Interface/Block}{Block of the normal derivative of the double layer integral operator}
\funByNameEven{BlockDnPlaneWave.m}{PreProcessing/IncidentWave/Block}{Block vector (=1 obstacle) of a the right hand side of the normal derivative of a plane wave}
\funByNameUneven{BlockDnPlaneWavePrecond.m}{PreProcessing/IncidentWave/Block}{Block vector (=1 obstacle) of a the right hand side of the normal derivative of a plane wave for the preconditioned problem of sound-hard scattering}
\funByNameEven{BlockDnPointSource.m}{PreProcessing/IncidentWave/Block}{Block vector (=1 obstacle) of a the right hand side of the normal derivative of a point source wave}
\funByNameUneven{BlockDnSingleLayer.m}{IntOperators/Dense/Interface/Block}{Block of the normal derivative of the single layer integral operator}
\funByNameEven{BlockDoubleLayer.m}{IntOperators/Dense/Interface/Block}{Block of the double layer integral operator}
\funByNameUneven{BlockIdentity.m}{IntOperators/Dense/Interface/Block}{Block of the identity integral operator}
\funByNameEven{BlockIncidentWave.m}{PreProcessing/IncidentWave}{Block vector of a generic incident wave (right hand side, block=1 obstacle)}
\funByNameUneven{BlockIntegralOperator.m}{IntOperators/Dense}{Generic dense block of an integral operator}
\funByNameEven{BlockPlaneWave.m}{PreProcessing/IncidentWave/Block}{Block vector (=1 obstacle) of a the right hand side of the trace of a plane wave}
\funByNameUneven{BlockPlaneWavePrecond.m}{PreProcessing/IncidentWave/Block}{Block vector (=1 obstacle) of a the right hand side of the trace of a plane wave for the preconditioned problem of sound-soft scattering}
\funByNameEven{BlockPointSource.m}{PreProcessing/IncidentWave/Block}{Block vector (=1 obstacle) of a the right hand side of the trace of a point source wave}
\funByNameUneven{BlockPotential.m}{PostProcessing/NearField/Functions}{Generic block potential matrix used to compute external or internal potentials}
\funByNameEven{BlockPrecondDirichlet.m}{IntOperators/Dense/Interface/Block}{Block of the preconditioned integral operator (sound soft case)}
\funByNameUneven{BlockPrecondNeumann.m}{IntOperators/Dense/Interface/Block}{Block of the preconditioned integral operator (sound hard case)}
\funByNameEven{BlockSingleLayer.m}{IntOperators/Dense/Interface/Block}{Block of the single layer integral operator}
\funByNameUneven{BoundaryOfObstacles.m}{PostProcessing/Geometry}{Extract the boundary of the obstacle}
\funByNameEven{CalderonProjector.m}{IntOperators/Dense/Interface/Full}{Full dense matrix of the Calderon projector}
\funByNameUneven{CheckPlacement.m}{PreProcessing/Geometry}{Verify if the obstacles are satisfying the condition (overlapping, \ldots)}
\funByNameEven{CreateRandomDisks.m}{PreProcessing/Geometry}{Place randomly random disks in a box}
\funByNameUneven{dbesselh.m}{Common}{First derivative of first kind Hankel function}
\funByNameEven{dbesselj.m}{Common}{First derivative of Bessel function}
\funByNameUneven{dbessely.m}{Common}{First derivative of Newton function}
\funByNameEven{DnDoubleLayer.m}{IntOperators/Dense/Interface/Full}{Full dense matrix of the normal derivative of the double layer integral operator}
\funByNameUneven{DnPlaneWave.m}{PreProcessing/IncidentWave/Full}{Full vector (=all obstacle) of a the right hand side of the normal derivative of a plane wave}
\funByNameEven{DnPlaneWavePrecond.m}{PreProcessing/IncidentWave/Full}{Full vector (=all obstacle) of a the right hand side of the normal derivative of a plane wave for the preconditioned problem of sound-hard scattering}
\funByNameUneven{DnPointSource.m}{PreProcessing/IncidentWave/Full}{Full vector (=all obstacle) of a the right hand side of the normal derivative of a point source wave}
\funByNameEven{DnSingleLayer.m}{IntOperators/Dense/Interface/Full}{Full dense matrix of the normal derivative of the single layer integral operator}
\funByNameUneven{DORT\_dielectric.m}{Examples/TimeReversal/FarField/NonPenetrable}{DORT for penetrable obstacles (far field)}
\funByNameEven{DORT\_NonPenetrable.m}{Examples/TimeReversal/FarField/NonPenetrable}{DORT for acoustic sound soft obstacles (far field)}
\funByNameUneven{DoubleLayer.m}{IntOperators/Dense/Interface/Full}{Full dense matrix of the double layer integral operator}
\funByNameEven{ExternalDoubleLayerPotential.m}{PostProcessing/NearField/Interface}{External potential of double layer potential only}
\funByNameUneven{ExternalPotential.m}{PostProcessing/NearField}{Compute potentials (single, double or linear combination) on a (Matlab) meshgrid and outside the obstacles}
\funByNameEven{ExternalSingleLayerPotential.m}{PostProcessing/NearField/Interface}{External potential of single layer potential only}
\funByNameUneven{fangle.m}{Common}{Angle with horizontal axis}
\funByNameEven{FarField.m}{PostProcessing/FarField}{Generic far field computation from densities}
\funByNameUneven{FarField\_to\_RCS.m}{PostProcessing/FarField}{Radar Cross Section (RCS) from far field}
\funByNameEven{FarFieldDoubleLayer.m}{PostProcessing/FarField/Interface}{Far field of the double layer potential only}
\funByNameUneven{FarFieldSingleLayer.m}{PostProcessing/FarField/Interface}{Far field of the single layer potential only}
\funByNameEven{FourierTruncation.m}{PreProcessing/Fourier}{Provide the number of mode to kept in the Fourier series}
\funByNameUneven{GetPotentialOptions.m}{PostProcessing/NearField/Functions}{Options for potential computations are condensed here}
\funByNameEven{HerglotzWave.m}{Examples/TimeReversal/FarField/Common}{Compute an Herglotz wave (linear combination of plane waves)}
\funByNameUneven{Identity.m}{IntOperators/Dense/Interface/Full}{Full dense matrix of the identity operator}
\funByNameEven{IncidentWave.m}{PreProcessing/IncidentWave}{Full vector of a generic incident wave (right hand side)}
\funByNameUneven{IncidentWaveOnGrid.m}{PostProcessing/IncidentWave}{Compute incident wave on a (Matlab) meshgrid}
\funByNameEven{IntegralOperator.m}{IntOperators/Dense}{Generic integral operator dense matrix (full)}
\funByNameUneven{InternalDoubleLayerPotential.m}{PostProcessing/NearField/Interface}{Internal potential of double layer potential only}
\funByNameEven{InternalPotential.m}{PostProcessing/NearField}{Compute potentials (single, double or linear combination) on a (Matlab) meshgrid and inside the obstacles}
\funByNameUneven{InternalSingleLayerPotential.m}{PostProcessing/NearField/Interface}{Internal potential of single layer potential only}
\funByNameEven{MaskMatrixObstacles.m}{PostProcessing/Geometry}{Matrix with boolean values inside or outside obstacles}
\funByNameUneven{PlaneWave.m}{PreProcessing/IncidentWave/Full}{Full vector (=all obstacle) of a the right hand side of the trace of a plane wave}
\funByNameEven{PlaneWavePrecond.m}{PreProcessing/IncidentWave/Full}{Full vector (=all obstacle) of a the right hand side of the trace of a plane wave for the preconditioned problem of sound-soft scattering}
\funByNameUneven{PlotCircles.m}{PostProcessing/Geometry}{Display obstacles on figure}
\funByNameEven{PointSource.m}{PreProcessing/IncidentWave/Full}{Full vector (=all obstacle) of a the right hand side of the trace of a point source wave}
\funByNameUneven{PrecondDirichlet.m}{IntOperators/Dense/Interface/Full}{Full dense matrix of the preconditioned integral operator (sound soft case)}
\funByNameEven{PrecondNeumann.m}{IntOperators/Dense/Interface/Full}{Full dense matrix of the preconditioned integral operator (sound hard case)}
\funByNameUneven{RCS.m}{PostProcessing/FarField}{Generic Radar Cross Section (RCS) computation from densities}
\funByNameEven{RCSDoubleLayer.m}{PostProcessing/FarField/Interface}{Radar Cross Section (RCS) of the double layer potential only}
\funByNameUneven{RCSSingleLayer.m}{PostProcessing/FarField/Interface}{Radar Cross Section (RCS) of the single layer potential only}
\funByNameEven{RectangularLattice.m}{PreProcessing/Geometry}{Build a rectangular lattice of disks}
\funByNameUneven{RemoveDisk.m}{PreProcessing/Geometry}{Remove some disks}
\funByNameEven{repeat\_horiz.m}{Common}{Copy/paste a row vector to build a matrix}
\funByNameUneven{repeat\_vert.m}{Common}{Copy/paste a column vector to build a matrix}
\funByNameEven{SingleLayer.m}{IntOperators/Dense/Interface/Full}{Full dense matrix of the single layer integral operator}
\funByNameUneven{SpAddIdentity.m}{IntOperators/Sparse/Functions}{Sparse function: add identity to a sparse operator}
\funByNameEven{SpBlockDnDoubleLayer.m}{IntOperators/Sparse/Interface/Block}{Sparse block of the normal derivative of the double layer integral operator}
\funByNameUneven{SpBlockDnSingleLayer.m}{IntOperators/Sparse/Interface/Block}{Sparse block of the normal derivative of the single layer integral operator}
\funByNameEven{SpBlockDoubleLayer.m}{IntOperators/Sparse/Interface/Block}{Sparse block of the double layer integral operator}
\funByNameUneven{SpBlockIdentity.m}{IntOperators/Sparse/Interface/Block}{Sparse block of the identity operator}
\funByNameEven{SpBlockIntegralOperator.m}{IntOperators/Sparse}{Generic sparse block of an integral operator}
\funByNameUneven{SpBlockIntegralOperator.m}{IntOperators/Sparse}{Generic integral operator sparse matrix (full)}
\funByNameEven{SpBlockPrecondDirichlet.m}{IntOperators/Sparse/Interface/Block}{Sparse block of the preconditioned integral operator (sound soft case)}
\funByNameUneven{SpBlockPrecondNeumann.m}{IntOperators/Sparse/Interface/Block}{Sparse block of the preconditioned integral operator (sound hard case)}
\funByNameEven{SpBlockSingleLayer.m}{IntOperators/Sparse/Interface/Block}{Sparse block of the single layer integral operator}
\funByNameUneven{SpDnDoubleLayer.m}{IntOperators/Sparse/Interface/Full}{Sparse matrix of the normal derivative of the double layer integral operator}
\funByNameEven{SpDnSingleLayer.m}{IntOperators/Sparse/Interface/Full}{Sparse matrix of the normal derivative of the single layer integral operator}
\funByNameUneven{SpDoubleLayer.m}{IntOperators/Sparse/Interface/Full}{Sparse matrix of the double layer integral operator}
\funByNameEven{SpIdentity.m}{IntOperators/Sparse/Interface/Full}{Sparse matrix of the identity operator}
\funByNameUneven{SpMatVec.m}{IntOperators/Sparse/Functions}{Sparse function: sparse matrix - vector product (possibly multiples)}
\funByNameEven{SpPrecondDirichlet.m}{IntOperators/Sparse/Interface/Full}{Sparse matrix of the preconditioned integral operator (sound soft case)}
\funByNameUneven{SpPrecondNeumann.m}{IntOperators/Sparse/Interface/Full}{Sparse matrix of the preconditioned integral operator (sound hard case)}
\funByNameEven{SpSingleLayer.m}{IntOperators/Sparse/Interface/Full}{Sparse matrix of the single layer integral operator}
\funByNameUneven{SpSingleMatVec.m}{IntOperators/Sparse/Functions}{Sparse function: sparse matrix - (only one) vector product}
\funByNameEven{TimeReversalOperator.m}{Examples/TimeReversal/FarField/Common}{Time reversal matrix in acoustic and far field context}
\funByNameUneven{TriangularLattice.m}{PreProcessing/Geometry}{Build a triangular lattice of disks}

\label{app:funFromName}
\chapter{List of \mudiff functions (folder name order)}\minitoc
\NewFolder{Common}
\funByFolderEven{repeat\_vert.m}{Copy/paste a column vector to build a matrix}
\funByFolderUneven{repeat\_horiz.m}{Copy/paste a row vector to build a matrix}
\funByFolderEven{fangle.m}{Angle with horizontal axis}
\funByFolderUneven{dbessely.m}{First derivative of Newton function}
\funByFolderEven{dbesselj.m}{First derivative of Bessel function}
\funByFolderUneven{dbesselh.m}{First derivative of first kind Hankel function}
\NewFolder{Examples/Benchmark}
\funByFolderEven{BenchmarkDirichlet.m}{Example of solution of sound soft scattering using different integral equations}
\funByFolderUneven{BenchmarkNeumann.m}{Example of solution of sound hard scattering using different integral equations}
\funByFolderEven{BenchmarkPenetrable.m}{Example of solution of penetrable scattering using single layer potential}
\funByFolderUneven{BenchmarkCalderon.m}{Example of solution of penetrable scattering using Calderon projectors}
\NewFolder{Examples/TimeReversal/FarField/Common}
\funByFolderEven{TimeReversalOperator.m}{Time reversal matrix in acoustic and far field context}
\funByFolderUneven{HerglotzWave.m}{Compute an Herglotz wave (linear combination of plane waves)}
\NewFolder{Examples/TimeReversal/FarField/NonPenetrable}
\funByFolderEven{DORT\_NonPenetrable.m}{DORT for acoustic sound soft obstacles (far field)}
\funByFolderUneven{DORT\_dielectric.m}{DORT for penetrable obstacles (far field)}
\NewFolder{IntOperators/Dense}
\funByFolderEven{BlockIntegralOperator.m}{Generic dense block of an integral operator}
\funByFolderUneven{IntegralOperator.m}{Generic integral operator dense matrix (full)}
\NewFolder{IntOperators/Dense/Interface/Block}
\funByFolderEven{BlockCalderonProjector.m}{Calderon projector block}
\funByFolderUneven{BlockDnDoubleLayer.m}{Block of the normal derivative of the double layer integral operator}
\funByFolderEven{BlockDnSingleLayer.m}{Block of the normal derivative of the single layer integral operator}
\funByFolderUneven{BlockDoubleLayer.m}{Block of the double layer integral operator}
\funByFolderEven{BlockIdentity.m}{Block of the identity integral operator}
\funByFolderUneven{BlockSingleLayer.m}{Block of the single layer integral operator}
\funByFolderEven{BlockPrecondNeumann.m}{Block of the preconditioned integral operator (sound hard case)}
\funByFolderUneven{BlockPrecondDirichlet.m}{Block of the preconditioned integral operator (sound soft case)}
\NewFolder{IntOperators/Dense/Interface/Full}
\funByFolderEven{CalderonProjector.m}{Full dense matrix of the Calderon projector}
\funByFolderUneven{SingleLayer.m}{Full dense matrix of the single layer integral operator}
\funByFolderEven{DnDoubleLayer.m}{Full dense matrix of the normal derivative of the double layer integral operator}
\funByFolderUneven{PrecondNeumann.m}{Full dense matrix of the preconditioned integral operator (sound hard case)}
\funByFolderEven{PrecondDirichlet.m}{Full dense matrix of the preconditioned integral operator (sound soft case)}
\funByFolderUneven{Identity.m}{Full dense matrix of the identity operator}
\funByFolderEven{DnSingleLayer.m}{Full dense matrix of the normal derivative of the single layer integral operator}
\funByFolderUneven{DoubleLayer.m}{Full dense matrix of the double layer integral operator}
\NewFolder{IntOperators/Sparse}
\funByFolderEven{SpBlockIntegralOperator.m}{Generic integral operator sparse matrix (full)}
\funByFolderUneven{SpBlockIntegralOperator.m}{Generic sparse block of an integral operator}
\NewFolder{IntOperators/Sparse/Functions}
\funByFolderEven{SpAddIdentity.m}{Sparse function: add identity to a sparse operator}
\funByFolderUneven{SpMatVec.m}{Sparse function: sparse matrix - vector product (possibly multiples)}
\funByFolderEven{SpSingleMatVec.m}{Sparse function: sparse matrix - (only one) vector product}
\NewFolder{IntOperators/Sparse/Interface/Block}
\funByFolderUneven{SpBlockPrecondDirichlet.m}{Sparse block of the preconditioned integral operator (sound soft case)}
\funByFolderEven{SpBlockSingleLayer.m}{Sparse block of the single layer integral operator}
\funByFolderUneven{SpBlockIdentity.m}{Sparse block of the identity operator}
\funByFolderEven{SpBlockDnDoubleLayer.m}{Sparse block of the normal derivative of the double layer integral operator}
\funByFolderUneven{SpBlockPrecondNeumann.m}{Sparse block of the preconditioned integral operator (sound hard case)}
\funByFolderEven{SpBlockDoubleLayer.m}{Sparse block of the double layer integral operator}
\funByFolderUneven{SpBlockDnSingleLayer.m}{Sparse block of the normal derivative of the single layer integral operator}
\NewFolder{IntOperators/Sparse/Interface/Full}
\funByFolderEven{SpDnSingleLayer.m}{Sparse matrix of the normal derivative of the single layer integral operator}
\funByFolderUneven{SpDoubleLayer.m}{Sparse matrix of the double layer integral operator}
\funByFolderEven{SpDnDoubleLayer.m}{Sparse matrix of the normal derivative of the double layer integral operator}
\funByFolderUneven{SpIdentity.m}{Sparse matrix of the identity operator}
\funByFolderEven{SpPrecondDirichlet.m}{Sparse matrix of the preconditioned integral operator (sound soft case)}
\funByFolderUneven{SpPrecondNeumann.m}{Sparse matrix of the preconditioned integral operator (sound hard case)}
\funByFolderEven{SpSingleLayer.m}{Sparse matrix of the single layer integral operator}
\NewFolder{PostProcessing/FarField}
\funByFolderUneven{RCS.m}{Generic Radar Cross Section (RCS) computation from densities}
\funByFolderEven{FarField\_to\_RCS.m}{Radar Cross Section (RCS) from far field}
\funByFolderUneven{FarField.m}{Generic far field computation from densities}
\NewFolder{PostProcessing/FarField/Interface}
\funByFolderEven{RCSDoubleLayer.m}{Radar Cross Section (RCS) of the double layer potential only}
\funByFolderUneven{RCSSingleLayer.m}{Radar Cross Section (RCS) of the single layer potential only}
\funByFolderEven{FarFieldSingleLayer.m}{Far field of the single layer potential only}
\funByFolderUneven{FarFieldDoubleLayer.m}{Far field of the double layer potential only}
\NewFolder{PostProcessing/Geometry}
\funByFolderEven{MaskMatrixObstacles.m}{Matrix with boolean values inside or outside obstacles}
\funByFolderUneven{PlotCircles.m}{Display obstacles on figure}
\funByFolderEven{BoundaryOfObstacles.m}{Extract the boundary of the obstacle}
\NewFolder{PostProcessing/IncidentWave}
\funByFolderUneven{IncidentWaveOnGrid.m}{Compute incident wave on a (Matlab) meshgrid}
\NewFolder{PostProcessing/NearField}
\funByFolderEven{InternalPotential.m}{Compute potentials (single, double or linear combination) on a (Matlab) meshgrid and inside the obstacles}
\funByFolderUneven{ExternalPotential.m}{Compute potentials (single, double or linear combination) on a (Matlab) meshgrid and outside the obstacles}
\NewFolder{PostProcessing/NearField/Functions}
\funByFolderEven{BlockPotential.m}{Generic block potential matrix used to compute external or internal potentials}
\funByFolderUneven{GetPotentialOptions.m}{Options for potential computations are condensed here}
\NewFolder{PostProcessing/NearField/Interface}
\funByFolderEven{InternalSingleLayerPotential.m}{Internal potential of single layer potential only}
\funByFolderUneven{InternalDoubleLayerPotential.m}{Internal potential of double layer potential only}
\funByFolderEven{ExternalSingleLayerPotential.m}{External potential of single layer potential only}
\funByFolderUneven{ExternalDoubleLayerPotential.m}{External potential of double layer potential only}
\NewFolder{PreProcessing/Fourier}
\funByFolderEven{FourierTruncation.m}{Provide the number of mode to kept in the Fourier series}
\NewFolder{PreProcessing/Geometry}
\funByFolderUneven{RemoveDisk.m}{Remove some disks}
\funByFolderEven{RectangularLattice.m}{Build a rectangular lattice of disks}
\funByFolderUneven{TriangularLattice.m}{Build a triangular lattice of disks}
\funByFolderEven{CreateRandomDisks.m}{Place randomly random disks in a box}
\funByFolderUneven{CheckPlacement.m}{Verify if the obstacles are satisfying the condition (overlapping, \ldots)}
\NewFolder{PreProcessing/IncidentWave}
\funByFolderEven{BlockIncidentWave.m}{Block vector of a generic incident wave (right hand side, block=1 obstacle)}
\funByFolderUneven{IncidentWave.m}{Full vector of a generic incident wave (right hand side)}
\NewFolder{PreProcessing/IncidentWave/Block}
\funByFolderEven{BlockPlaneWave.m}{Block vector (=1 obstacle) of a the right hand side of the trace of a plane wave}
\funByFolderUneven{BlockDnPlaneWave.m}{Block vector (=1 obstacle) of a the right hand side of the normal derivative of a plane wave}
\funByFolderEven{BlockPlaneWavePrecond.m}{Block vector (=1 obstacle) of a the right hand side of the trace of a plane wave for the preconditioned problem of sound-soft scattering}
\funByFolderUneven{BlockDnPlaneWavePrecond.m}{Block vector (=1 obstacle) of a the right hand side of the normal derivative of a plane wave for the preconditioned problem of sound-hard scattering}
\funByFolderEven{BlockDnPointSource.m}{Block vector (=1 obstacle) of a the right hand side of the normal derivative of a point source wave}
\funByFolderUneven{BlockPointSource.m}{Block vector (=1 obstacle) of a the right hand side of the trace of a point source wave}
\NewFolder{PreProcessing/IncidentWave/Full}
\funByFolderEven{PointSource.m}{Full vector (=all obstacle) of a the right hand side of the trace of a point source wave}
\funByFolderUneven{PlaneWavePrecond.m}{Full vector (=all obstacle) of a the right hand side of the trace of a plane wave for the preconditioned problem of sound-soft scattering}
\funByFolderEven{PlaneWave.m}{Full vector (=all obstacle) of a the right hand side of the trace of a plane wave}
\funByFolderUneven{DnPointSource.m}{Full vector (=all obstacle) of a the right hand side of the normal derivative of a point source wave}
\funByFolderEven{DnPlaneWavePrecond.m}{Full vector (=all obstacle) of a the right hand side of the normal derivative of a plane wave for the preconditioned problem of sound-hard scattering}
\funByFolderUneven{DnPlaneWave.m}{Full vector (=all obstacle) of a the right hand side of the normal derivative of a plane wave}

\label{app:funFromFolder}



\bibliographystyle{plain}
\bibliography{Biblio_mudiff.bib}



\end{document}

