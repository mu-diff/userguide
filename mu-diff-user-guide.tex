\documentclass[11pt]{book}

\usepackage{amsmath}
%\usepackage{showkeys}
\usepackage{amsfonts}
\usepackage{amssymb}
\usepackage{color,graphicx}
\usepackage{epstopdf}
\usepackage{graphics}
\usepackage{subfigure}

\usepackage{pdfsync}

\usepackage{hyperref}
\textwidth 17cm
\textheight 21cm
\setlength{\oddsidemargin}{-1.5mm}
\setlength{\evensidemargin}{-1.5mm}


\usepackage{fullpage}
\usepackage{xspace}

\newcommand{\ui}{u_{I}}
\newcommand{\ud}{u_{D}}

\newcommand{\uif}{u_{I}^f}
\newcommand{\udf}{u_{D}^f}
\newcommand{\inc}{\textrm{inc}}

\newcommand{\uii}{u_{I}(\cdot\,;\xx_{i})}
\newcommand{\udi}{u_{D}(\cdot\,;\xx_{i})}
\newcommand{\uij}{u_{I}(\cdot\,;\xx_{j})}
\newcommand{\udj}{u_{D}(\cdot\,;\xx_{j})}

\newcommand{\f}{{\rm f}}
\newcommand{\vrm}{{\rm v}}
\newcommand{\urm}{{\rm u}}


\usepackage{MesCommandes}


% Fontes maths
\usepackage{mathrsfs}        % pour \mathscr{F}, W...
\usepackage{amsmath}
\usepackage{amsfonts}
\usepackage{amssymb}
\usepackage{stmaryrd}        % pour \llbracket
\usepackage{array}           % d�f largeur colonne et justif, ou mode maths
\usepackage{epsfig}

\usepackage{color}
\usepackage{amsthm}
\usepackage{amsmath}
\usepackage{amsfonts}
\usepackage{amssymb}
%\usepackage{showkeys}
\usepackage{graphicx}
\usepackage{graphics}
\usepackage{subfigure}

\usepackage{import} % pour les .pdf_tex

\usepackage{pdfsync}


% ------------------------
% Environnements maths
\usepackage{amsthm}          % pour le style des th�or�mes /preuves

\theoremstyle{plain} 
\newtheorem{thm}{Theorem}
\newtheorem{prop}{Proposition}  % meme compteur que les theoremes
\newtheorem{lem}{Lemma}     % numerotation repart a zero a chaque chapitre
\newtheorem{cor}{Corollary}
\newtheorem{defn}{Definition}
%\theoremstyle{definition}
\newtheorem{pro}{Property}
%\theoremstyle{remark}{remark}
\newtheorem{rmk}{Remark}
\newtheorem{appl}{Application}
\newtheorem{ex}{Example}
%\renewcommand{\proofname}{Proof}
\newenvironment{disarray}%
{\everymath{\displaystyle\everymath{}}\array}%
{\endarray}
\makeatletter
\newenvironment{myalign}{\@fleqntrue\align}{\endalign\@fleqnfalse}
\makeatother 
\makeatletter
\newenvironment{myalign*}{%
  \@fleqntrue\csname align*\endcsname
  }{%
  \csname endalign*\endcsname
  } 
\newcommand{\thmcaption}[1]{\addcontentsline{lthm}{theoreme}{#1}}
\makeatletter
\let\l@theoreme\l@figure
\makeatother
\makeatletter
\newcommand{\listoftheorems}{%
\chapter*{Liste des th�or�mes et propositions}\@starttoc{lthm}}
\makeatother
% ------------------------


% ------------------------
% Op�rateurs maths
\usepackage{amsopn}          % nouvelles fonctions maths tq \sin

\DeclareMathOperator*{\argsinh}{argsinh}
\DeclareMathOperator*{\sech}{sech}
\DeclareMathOperator*{\oo}{\mathit{o}}       % notations de Landau (indic�es)
% Symboles pour les matrices rectangulaires, horizontales et verticales
\newcommand\rectvert{\gensymbole{6}{3}}
\newcommand\recthorz{\gensymbole{3}{6}}
\newcommand\gensymbole[2]{{\unitlength=1pt
\begin{picture}(#1,#1)
\put(0,0){\line(1,0){#2}}
\put(#2,0){\line(0,1){#1}}
\put(0,0){\line(0,1){#1}}
\put(0,#1){\line(1,0){#2}}
\end{picture}}}



\newcommand{\ds}{\displaystyle}
\newcommand{\N}{\mathbb{N}}
\newcommand{\R}{\mathbb{R}}
\newcommand{\C}{\mathbb{C}}
%
\newcommand{\esp}{\makebox[2\width][l]{\quad}}          % algorithmes
% ------------------------

% Couleurs
\usepackage{xcolor}
\newcommand{\modified}[1]{{\color{red}   #1  }}

% Largeurs de figures
\def\MiniLength{0.32\textwidth}
\def\MiniLengthb{0.46\textwidth}
\def\MiniLengthc{0.57\textwidth}
\newcommand{\larg}{10cm}        % largeur d'une figure seule
\newcommand{\largeurfig}{0.48\textwidth}  % largeur de 2 figures cote a cote
\newcommand{\largtrois}{0.33\textwidth}  % largeur de  figures cote a cote



\title{$\mu$-diff Matlab toolbox: user guide}

%\author{X. Antoine\thanks{Institut Elie Cartan Nancy (IECN), Nancy University, INRIA Corida Team, B.P. 239, F-54506
%    Vandoeuvre-l\`es-Nancy Cedex, France.}, \and
%  C. Besse\thanks{?} \and P. Klein}

\author{Bertrand Thierry\footnotemark[1] , Xavier Antoine\footnotemark[2] , Chokri Chniti\footnotemark[3]   and 
Hasan Alzubaidi\footnotemark[3]}

\date{}

\begin{document}

\maketitle

\renewcommand{\thefootnote}{\fnsymbol{footnote}}

%\footnotetext[1]{Laboratoire J.L. Lions (LJLL), University of Paris VI, Paris, France. ({\tt bertrand.thierry1@gmail.com}).}
%\footnotetext[2]{Universit\'e de Lorraine, Institut Elie Cartan de Lorraine, UMR 7502, Vandoeuvre-l\`es-Nancy, F-54506, France. ({\tt xavier.antoine@univ-lorraine.fr}).}
%\footnotetext[3]{Department of Mathematics, University College in Qunfudah, Umm Al-Qura University, Saudi Arabia.
% ({\tt cachniti@uqu.edu.sa;hmzubaidi@uqu.edu.sa}).}


\renewcommand{\thefootnote}{\arabic{footnote}}

\tableofcontents

\mtcaddchapter[Copyright]                          % solution pour minitoc
\markboth{\uppercase{Copyright}}{\uppercase{Copyright}} 

\ldots 

\chapter{Introduction}
\mtcaddchapter[Introduction]                          % solution pour minitoc
\markboth{\uppercase{Introduction}}{\uppercase{Introduction}} 

\ldots 

\chapter{Mathematical background}
The \mudiff toolbox is based on integral equation and, in fact, provides the the four classical boundary integral operators. The derivation and the choice of the integral formulation is let to the user, even if some of them are given in this chapter, which contains all the necessary mathematical background to solve  time-harmonic wave scattering problems by disks, by penetrable or impenetrable obstacles. The only limitation is the integral formulation: if the integral formulation is written, then it can be solved using \mudiff.

This chapter begins by presenting the potential theory and the four classical boundary integral operators with their main properties. The case of the scattering by disks is then studied and the boundary integral operators are projected in the Fourier bases, leading to infinite matrices but with analytic expression of the coefficients. The case of the truncation is then discussed, and the chapter concludes with the expression of both the near- and far-field, and the projections of the right-hand side on the Fourier bases (incident wave).

\chapter{Toolbox}
\section{Generalities}
The authors try to keep the notations between the mathematical background and the \matlab toolbox. In particular, lots of arguments are the same in the different functions and they are here presented once for all. A manual of every function is moreover provided by simply typing in the \matlab command window 
\begin{verbatim}
help name_of_the_function
\end{verbatim}
In all what follows, and if nothing is specified, then the following arguments refer to the below ones. The indices $p$ and $q$ vary from $1$ to $\Nscat$ where $\Nscat$ is the number of obstacles.

\paragraph{Geometry:}

\begin{center}
%\rowcolors{1}{white}{gray}
\begin{tabular}{|c |c | p{10cm}|}
\hline Name & Size or Type & Content\\[0.2cm]\hline\hline
\code{N\_scat} & $[1\times 1]$ & Number of obstacles $\Nscat$.\\\hline
\code{O} & $[2\times \Nscat]$ & Matrix of the centers of the disks such that \code{O(1,p)} is the x-coordinate of the $p^{th}$ obstacle.\\\hline
\code{Op} & $[2\times 1]$ & Coordinate of the $p^{th}$ scatterer.\\\hline
\code{Oq} & $[2\times 1]$ & Coordinate of the $q^{th}$ scatterer.\\\hline
\code{a} & $[1\times \Nscat]$ & Vector of the radii of the disks such that \code{a(p)} is the radius of the $p^{th}$ scatterer.\\\hline
\code{ap} & $[1\times 1]$ & Radius of the $p^{th}$ scatterer.\\\hline
\code{aq} & $[1\times 1]$ & Radius of the $q^{th}$ scatterer.\\\hline
\end{tabular}
\end{center}

\paragraph{Parameters (wavenumbers, incident waves, fourier series,\ldots):}
\begin{center}
\begin{tabular}{|c |c | p{10cm}|}
\hline Name & Size or Type & Content\\[0.2cm]\hline\hline
\code{beta\_inc} & $[1\times 1]$ & Angle of direction of a plane wave $e^{ik (\cos(\beta)x + \sin(\beta)y)}$.\\\hline
\code{XS} & $[2\times 1]$ & Center $(x_s,y_s)$ of a point source: $x_s=$\code{XS(1)} and $y_s=$\code{XS(2)}. A point source wave is then given by $i/4\Hz(k\|\xx-\xx_s\|)$, with $\xx=(x,y)$ and $\xx_s=(x_s,y_s)$ and $\Hz$ the zeroth order Hankel function of first kind.\\\hline
\code{k} & $[1\times 1]$ & Wavenumber $k$ in the vacuum.\\\hline
\code{k\_int} & $[1\times \Nscat]$ & Wavenumber in the obstacles: $\kintp=$\code{k\_int(p)}. If \code{k\_int} is a scalar then $\kintp$ = \code{k\_int} for all $p$.\\\hline
\code{M\_modes} & $[1\times \Nscat]$ & Vector of index of truncation of the Fourier series, \ie \code{M\_modes(p)}=$\Np$\\\hline
\code{Np} & $[1\times 1]$ & Corresponds to $\Np$, the truncation index in the Fourier serie.\\\hline
\code{Nq} & $[1\times 1]$ & Corresponds to $\Nq$, the truncation index in the Fourier serie.\\\hline
\end{tabular}
\end{center}

\paragraph{Incident wave:}
\begin{center}
\begin{tabular}{|c |c | p{10cm}|}
\hline Name & Size or Type & Content\\[0.2cm]\hline\hline
\code{PlaneWave} & $[1\times 1]$ & Wavenumber in the vacuum.\\\hline
\end{tabular}
\end{center}

\paragraph{Integral operators:} they are indexed and named by the following table:
\begin{center}
\begin{tabular}{|c |c | c | p{9cm}|}
\hline Index & Letter & Operator & \mudiff abreviation \\\hline\hline
0 & - & - & Null operator \\
1 & $I$ & \code{Identity} & Identity\\
2 & $L$ & \code{SingleLayer} & Trace of the single-layer\\
3 & $M$ & \code{DoubleLayer} & Trace of the double-layer\\
4 & $N$  & \code{DnSingleLayer}& Normal derivative of the single-Layer\\
5 & $D$  & \code{DnDoubleLayer}&Normal derivative of the double-Layer\\ 
6 & $\hat{L}$  & \code{PrecondDirichlet}& Single-scattering preconditioned trace of the single-layer operator\\
7 & $\hat{D}$  & \code{PrecondNeumann}& Single-scattering preconditioned normal derivative trace of the double-layer operator\\ \hline
\end{tabular}
\end{center}


\section{Pre-Processing}

The pre-processing functions are located in the \code{PreProcessing} folder, which is divided in two parts: the construction of the obstacles in \code{Geometry}, and the computation of the right-hand side, \ie the incident waves on the obstacles, in \code{IncidentWave}.

\subsection{Creating Obstacles}

\subsubsection{Manual placement}
The disks can be created manually by simply creating the two variables \code{O} and \code{a} containing respectively the coordinates of the disks and their radii. For example, for three obstacles placed on $(-1,2)$, $(5,5)$ and $(-15,10)$ with radii $0.1$, $0.5$ and $10$:
\begin{verbatim}
O = [-1, 5, 2 ; -15, 5, 10];
a = [0.1, 0.5, 10];
\end{verbatim}

\subsubsection{Periodic placement}

Two build-in functions are available with the toolbox to create disks periodically, with a rectangular or a triangular lattice, as shown on figure \ref{??}. The two functions are called as follows, for the rectangular lattice:
\begin{verbatim}
O = RectangularLattice(bx, by, Nx, Ny);
\end{verbatim}
and for the triangular lattice:
\begin{verbatim}
O = TriangularLattice(bx,by,Nx,Ny);
\end{verbatim}
where :
\begin{itemize}
\item \code{bx} and \code{by} are the distance separating two centers in the $x$ coordinate and \code{by} the distance between two rows in the $y$-direction. The two other parameter 
\item \code{Nx}: number of disks in a row
\item \code{Ny}: number of rows
\end{itemize}
For both functions, the vector of radii must be built separately. If the disks are unitary then the following command can be used
\begin{verbatim}
a = ones(size(O,2));
\end{verbatim}


\subsubsection{Random placement}

Finally, to place randomly obstacles in a box $[\code{xmin}, \code{xmax}]\times[\code{ymin},\code{ymax}]$ with also a random radius, a function \code{CreateRandomDisks} is very helpful. In its simplest version, the function is called as:
\begin{verbatim}
[O, a] = CreateRandomDisks(xmin, xmax, ymin, ymax, N_scat);
\end{verbatim}
The function will then create \code{N\_scat} disk with unit radius in the desired box. This is however probably too simple and the function can be called in a more complex form:
\begin{verbatim}
[O, a] = CreateRandomDisks(xmin, xmax, ymin, ymax, N_scat, 
           amin, amax, dmim, dmax, O_avoid, a_avoid, dmin_avoid, dmax_avoid);
\end{verbatim}
where
\begin{center}
\begin{tabular}{|c |c|c | p{10cm}|}
\hline Variable & Type & Default & Action\\\hline
\code{amin} & scalar  & 1 & Minimal (random) radius of the obstacles allowed \\\hline
\code{amax} & scalar  & 1 & Maximal (random) radius of the obstacles  allowed\\\hline
\code{dmin} & scalar & \code{realmin} & Minimal distance allowed between two obstacle (not between the centers!). Setting $\leq 0$  value will set \code{dmin} to \code{realmin} (\ie ignore it)\\\hline
\code{dmax} & scalar & \code{realmax} & Maximal distance allowed between two obstacle (not between the centers!). The maximal distance is quickly reached! Setting $\leq 0$  value will set \code{dmax} to \code{realmax} (\ie ignore it).\\\hline
\code{O\_avoid} & \code{[2 x N]} & \code{[]} & Center of \code{N} hole(s) where the obstacles must not overlap. Usefull for example for point source location.\\\hline
\code{a\_avoid} & \code{[1 x N]} & \code{[]} & Radii of the \code{N} holes\\\hline
\code{dmin\_avoid} & \code{[1 x N]} & \code{[]} & Minimal distance between an obstacle and a hole\\\hline
\end{tabular}
\end{center}
These optional arguments are not mandatory in the function call. For example:
\begin{verbatim}
[O, a] = CreateRandomDisks(xmin, xmax, ymin, ymax, N_scat, amin, amax);
\end{verbatim}
will create random disks with random radii without taking into account the distances between the disks (except obviously the overlapping).

For example, building 7 obstacles in the box $[-10,10]\times[-10,10]$ with radii between $0.1$ and $0.5$. The disks must be separated at minimum by a distance of $0.1$ and without maximum value. The command is then:
\begin{verbatim}
[O, a] = CreateRandomDisks(-10, 10, -10, 10, 7, 0.1, 0.5, 0.1, -1);
\end{verbatim}
Now imagine that a point source is located on $(2,2)$ and that the obstacles must be separated from the source from at least $0.3$, then the ``\code{*\_avoid}'' arguments can be used and command can be
\begin{verbatim}
[O, a] = CreateRandomDisks(-10, 10, -10, 10, 7, 0.1, 0.5, 0.1, -1, [2;2], 0.3);
\end{verbatim}
the disk centered on $(2,2)$ with radius $0.3$ will then be avoided. A second option is to set \code{a\_void} to zero and set the minimal distance \code{dmin\_avoid} to $0.3$:
\begin{verbatim}
[O, a] = CreateRandomDisks(-10, 10, -10, 10, 7, 0.1, 0.5, 0.1, -1, [2;2], 0, 0.3);
\end{verbatim}


\begin{remark}
To verify if a disk is well placed, \code{CreateRandomDisks} calls \code{CheckPlacement} function, which can also be useful for a user placing obstacles.
\end{remark}

\subsubsection{Removing disks}

The function \code{RemoveDisk} aims to remove some disks of the geometrical configuration, either disk by disk, by row or by column or by radius. Here is its syntax
\begin{verbatim}
[O,a] = RemoveDisk(O_old, a_old, ...);
\end{verbatim}
where \code{O\_old} and \code{a\_old} are the centers and radii of the current geometry. Without optional argument, the function will return \code{[O\_old, a\_old]} and the available arguments are:
\begin{itemize}
\item \code{[O,a] = RemoveDisk(..., 'X', [X1, X2, ..., XN]);}\\
Remove all the points with X abscissa X1, X2, ..., or XN
\item \code{[O,a] = RemoveDisk(...,  'Y', [Y1, Y2, ..., YN]);}\\
Remove all the points with Y ordinate Y1, Y2, ..., or YN
\item \code{[O,a] = RemoveDisk(..., 'XY', [[X1;Y1], [X2;Y2], ..., [XN;YN]]);}\\
Remove all the points [X1;Y1], [X2;Y2], ..., and [XN;YN]
\item \code{[O,a] = RemoveDisk(..., 'Radius', [a1, a2, ..., aN]);}\\
Remove all the disk with radius a1, a2, ..., or aN
\item \code{[O,a] = RemoveDisk(..., 'Verbosity', VERBOSITY);}\\
set VERBOSITY to 0 to avoid display message, to 1 to only show results, and to $>1$ to see everything (default).
\end{itemize}

For example, to remove every obstacle on the row of $x-$abscissa $1$ and $y-$ordinate $2.5$:\\
\code{  [O,a] = RemoveDisk(O\_old, a\_old, 'X', 1, 'Y', 2.5);}\\
Or, to remove the obstacles centered on $(2,5)$ and (3,4):\\
\code{  [O,a] = RemoveDisk(O\_old, a\_old, 'XY', [2,3;5,4]);}\\

\subsection{Incident waves}

Two different incident waves, plane wave and point source wave, are available in the \mudiff toolbox, but it should be highlighted that the user can build his/her own incident wave. A right-hand side $b$ is decomposed by blocks, each of these representing one obstacle: $b= (b_p)_{p=1,\ldots,\Nscat}$. As a different condition can be applied on an obstacle or a different integral equation can be considered on it, each block can be specified separately. Depending on the type of desired right-hand side, the block function \code{BlockIncidentWave} will compute the vector $b_p$. The frontal function \code{IncidentWave}, which compute the whole vector $b$, consists on calling  \code{BlockIncidentWave} for each obstacle and assembling the vector. Interfaced functions are available to compute directly some incident wave, such as \code{PlaneWave}, \code{PointSource} or \code{DnPlaneWave} and \code{DnPointSource}. They call \code{IncidentWave} with the right argument. There is no need here to described these functions, however their help is helpful and contains the mathematical description of the incident wave.

\subsubsection{\code{BlockIncidentWave}}







\section{Integral operators}

Two different type of storage are provided with the \mudiff toolbox: dense and sparse. The dense version store the whole matrix in memory whereas the sparse version uses the special structure of the matrix of an integral operator to store it. The sparse storage in \mudiff and this user guide has nothing to deal with the sparse storage provided in \matlab such as \texttt{sparse} function. The dense storage is easier to use and works pretty well for small scale problems. It also presents the advantage of providing the whole matrix of the integral operator, which can be useful for spectrum analysis for example. On the other hand, for a large number of circular obstacles and/or for large frequency, the memory storage becomes too important and the sparse version must be used. One should be however careful: the sparse matrix-vector product, based on the cross-correlation (\texttt{xcorr} \matlab function), is very sensitive to the number of modes chosen in the truncation of the Fourier series. Indeed, if too many modes are kept, the matrix-vector product show to be unstable. The formula (\ref{??}) seems to provide stability.

For both dense and sparse case, let $\Ab$ be a generic matrix representing the matrix of one of the four boundary integral operator, $\Lb,\Mb,\Nb$ and $\Db$. As highlighted in previous chapter, $\Ab$ has the following structure, for $p,q=1,\ldots,\Nscat$ and $p\neq q$:
\begin{itemize}
\item $\Abpp$ is diagonal.
\item $\Abpq$ is full and can be divided as $\Abpq = \AbpqL\Tbpq\AbpqR$ where $\AbpqL$ and $\AbpqR$ are diagonal and called respectively the left and right part, and $\Tbpq = (\Tbpqmn))$, with $\Tbpqmn = i\pi e^{??}H_0^{(1)}(k\bpq)$, is a Toeplitz matrix.
\end{itemize}
In the sparse version, diagonal submatrices $\Abpp,\AbpqL$ and $\AbpqR$  are stored as a vector of size respectively $2\Np+1$, $2\Np+1$ and $2\Nq+1$, and the Toeplitz matrices $\Tbpq$ are stored as vectors of size $2\Np+2\Nq-1$.

This section is naturally divided in two part, the first being devoted to the dense storage and the second to the sparse version.

\subsection{Available integral operators}

The integral operators are numbered as follows
\begin{enumerate}
\item Null operator
\end{enumerate}


\subsection{Dense storage}
\subsubsection{Block structures}

A matrix $\Ab$ is created by blocks using \texttt{BlockIntegralOperator} function, which has the following syntax
\begin{verbatim}
	Apq = BlockIntegralOperator(Op, ap, Np, Oq, aq, Nq, k, TypeOfOperator)
\end{verbatim}
where:
\begin{itemize}
\item \texttt{Op}(resp. \texttt{Oq}): \texttt{2$\times$1}, vector containing the centers of the disk such that \texttt{Op(1,p)}$=x_{\OOp}$ and \texttt{Op(2,p)}$=y_{\OOp}$ (resp. \texttt{Oq(1)}$=x_{\OOq}$ and \texttt{Oq(2)}$=y_{\OOq}$)
\item \texttt{ap}(resp. \texttt{aq}): \texttt{1$\times$1}, Radius of the disks $\OOp$ (resp. $\OOq$)
\item \texttt{Np} (resp. \texttt{Nq}): \texttt{1$\times$1}, Index of truncation $\Np$ (resp. $\Nq$)
\item \texttt{k}: \texttt{1$\times$1}, Wavenumber in the vacuum
\item \texttt{TypeOfOperator}: \texttt{Variable size}, Type of integral operator to be computed.
\end{itemize}

\subsubsection{Full matrix}

\subsection{Sparse storage}
\subsubsection{Block structures}
\subsubsection{Full matrix}

\section{Post-Processing}
\subsubsection{Near field}
\subsubsection{Far field and Radar Cross Section (RCS)}
\subsubsection{Plot and display}



\chapter{Examples}
The aim of this chapter is to provide some examples of multiple scattering problems solved by  the \mudiff toolbox.
The impenetrable case with a Dirichlet, a Neumann or a mixed of both boundary conditions  set on the boundaries of the obstacles are
fully treated.


\section{The Dirichlet boundary-value problem}

Let us consider the scattering problem by a collection of sound-soft obstacles
$$
\left\{\begin{array}{r c l l}
(\Delta +k^2)u & = & 0, & \text{ in }\Omegaps,\\
u & = & -\uinc, & \text{ on }\Gamma,\\
\multicolumn{4}{l}{\qquad \qquad u \text{ outgoing},}
\end{array}\right.
$$
with  $\Omegam = \bigcup_{p=1}^{M}\Omegamp$.
We propose to solve this problem through various integral equations: the EFIE (\ref{eq:EFIE}), the MFIE (\ref{eq:MFIE}), the 
CFIE (\ref{eq:CFIE}) and the single-scattering preconditioned integral equations (\ref{eqEqINt:LAsglLA}). We show how
 to use both the full and  sparse storages of the matrices.
Before starting, we recommend to use the  single-scattering preconditioned integral equation as presented in \S\ref{secEx:PrecondD}. 
Indeed, the resulting system is well-posed and is well-conditioned leading to an efficient solution by a Krylov subspace iterative solver.

\subsection{Pre-processing}

Let us first consider a collection of three sound-soft unit circular cylinders. The  wavenumber is 
$k=2\pi$ and the direction of  incidence of the
 wave is $\beta = 0$ degree. The resulting  \mudiff pre-processing code for setting  these parameters is then
\begin{lstlisting}
%% Pre-processing
% Three unit disks 
O = [-5, 0, 5; -2, 0, 2];
a = [1, 1, 1];
%Set the parameters...
k = 1; %wavenumber
beta_inc = 0; %incident angle
%Fourier series truncation parameter
M_modes = FourierTruncation(a, k, 'Min', 1);
\end{lstlisting}

For each integral equation, we now present the assembly process, the computation of the solution and finally the post-processing of the computed wave
fields. 
The common pre-processing part is the one described above. All the functionalities presented here are also available in  the
 file \texttt{BenchmarkDirichlet.m} which is located in the  \folder{Examples/Benchmark} folder.
 To launch the computations, it is sufficient to type the following command
\begin{lstlisting}
BenchmarkDirichlet;
\end{lstlisting}
in the Matlab window once $\mu$-diff has been installed.

\subsection{The case of the EFIE}


This integral formulation reads as
$$
\left\{\begin{array}{r c l}
u &=& \Lop\rho,\\
L\rho &=& -\uinc|_{\Gamma},
\end{array}\right.
$$
where the first line is the integral equation representation of the exterior wavefield $u$ and the second one is  the surface 
integral equation to solve. 

\subsubsection{Dense storage}

In \mudiff, the surface single-layer operator $L$ and the incident plane wave field $\uinc|_{\Gamma}$ are  predefined quantities.
If the full storage of the integral equation is used, then the direct solution of the resulting linear system can be
obtained by the standard backslash Matlab operator $\backslash$ 
\begin{lstlisting}
%Right-hand side
Uinc = PlaneWave(O, a, M_modes, k, beta_inc);
%% Assembling
%Matrix of the system (the two following lines are the same)
L = SingleLayer(O, a, M_modes, k);
%% Solving (here, direct)
rho = L \ Uinc;
\end{lstlisting}
\medskip

Once the surface wavefield has been computed, the RCS can be calculated by the following $\mu$-diff  commands
\begin{lstlisting}
%% Post-processing
%Scattering angles 
theta_RCS = 0:360;
theta_RCS_rad = theta_RCS*2*pi/360;
%Radar Cross Section for the single-layer representation (<-> [1,0])
myRCS = RCS(O, a, M_modes, k, theta_RCS_rad, rho, [1,0]);
plot(theta_RCS, myRCS, 'k');
\end{lstlisting}

\subsubsection{Sparse storage}

For the sparse storage version, only the assembly process of the single-layer matrix and the system solution
need to be modified as follows
\begin{lstlisting}
%Matrix of the system (the two following lines are the same)
SpL = SpSingleLayer(O, a, M_modes, k);
%% Solving (here, direct)
rho = gmres(@(X)SpMatVec(X, M_modes, SpL), Uinc);
\end{lstlisting}
\medskip

\subsection{The case of the MFIE}


The resolution of the scattering problem by the MFIE (\ref{eq:MFIE}) leads to the integral equation representations
$$
\left\{\begin{array}{r c l}
u &=& \Lop\rho,\\
\dsp \left(\frac{I}{2}+N\right)\rho &=& -\dn\uinc|_{\Gamma}.
\end{array}\right.
$$

\subsubsection{Dense storage}

The MFIE operator $$\left(\frac{I}{2}+N\right)$$ can be computed  thanks to the frontal function \IntegralOperator with two
 arguments: the type of the operators (for the identity operator and the double-layer potential operator $N$, see
  Table \ref{table:IntOp}) and their associated weights ($0.5$ and $1$). 
\begin{lstlisting}
%Right hand side
DnUinc = DnPlaneWave(O, a, M_modes, k, beta_inc);
%% Assembling
%Matrix of the system (the two following lines are the same)
A_MFIE = IntegralOperator(O, a, M_modes, k, [1, 4], [0.5, 1]);
%% Solving (here, direct)
rho = A_MFIE \ DnUinc;
\end{lstlisting}
\medskip

The post-processing part is exactly the same as for the EFIE since the surface equation is based on the 
volume single-layer integral representation.

\subsubsection{Sparse storage}

The sparse storage version is almost the same as for the dense storage except for  assembling  the matrix and solving the linear system. Indeed, the matrices
 $I$ and $N$ cannot be computed by the same function since the sparse  function representations \SpIntegralOperator and
  \IntegralOperator cannot be  summed together. It is however possible to add the identity to a ''sparse operator'' thanks to \SpAddIdentity
\begin{lstlisting}
SpN = SpDnSingleLayer(O, a, M_modes, k);
%Add I/2 to N:
SpA_MFIE = SpAddIdentity(SpN, 0.5, M_modes)
rho = gmres(@(X)SpMatVec(X, M_modes, SpA_MFIE), DnUinc);
\end{lstlisting}
\medskip

\subsection{The case of the CFIE}

Let us now consider  the well-posed and well-conditioned CFIE (see also Eq. (\ref{eq:CFIE}))
$$
\left\{\begin{array}{r c l}
u &=& \Lop\rho,\\
\dsp \left[\alpha\eta L  + (1-\alpha)\left(\frac{I}{2}+N\right)\right]\rho &=& -\alpha\eta\uinc|_{\Gamma} - (1-\alpha) \dn\uinc|_{\Gamma}.
\end{array}\right.
$$
Here, we fix the parameters  to $\alpha=0.5$ and $\eta = i/k$. 

\subsubsection{Dense storage}

The operator $$(1-\alpha)\left(\frac{I}{2}+N\right) + \alpha\eta$$ is computed in \mudiff by
using the \IntegralOperator function,  the post-processing remaining  unchanged,
\begin{lstlisting}
%CFIE
alpha = 0.5;
eta = i/k;
%Right-hand side
Uinc = PlaneWave(O, a, M_modes, k, beta_inc);
DnUinc = DnPlaneWave(O, a, M_modes, k, beta_inc);
BCFIE = alpha*eta*Uinc + (1-alpha)*DnUinc;
%% Assembling
%Matrix of the system (the two following lines are the same)
ACFIE = IntegralOperator(O, a, M_modes, k, [2, 1, 4], [alpha*eta, 0.5*(1-alpha), 1-alpha]);
%% Solving (here, direct)
rho = ACFIE \ BCFIE;
\end{lstlisting}
\medskip

\subsubsection{Sparse storage}

The sparse storage version changes compared to the dense one: the operators $I/2 + N$ and $L$ are computed separately and merged during the matrix-vector products. This is done in the \SpMatVec function
\begin{lstlisting}
SpL = SpSingleLayer(O, a, M_modes, k);
SpN = SpDnSingleLayer(O, a, M_modes, k);
SpA_MFIE = SpAddIdentity(SpN, 0.5, M_modes)
%% Solving and combining operators:
rho = gmres(@(X)SpMatVec(X,M_modes,{SpL, SpA_MFIE}, [alpha*eta, 1-alpha]), B_CFIE);
\end{lstlisting}
\medskip

\subsection{The case of the single-scattering preconditioned integral equation}
\label{secEx:PrecondD}
We strongly recommend
 to use the single-scattering preconditioned version of the EFIE, which is rigorously the same as the MFIE and CFIE and, up to an invertible operator, 
 to any other boundary integral equation (see Proposition \ref{prop:SingleScat}). The EFIE version is available in \mudiff and is represented as
$$
\begin{cases}
u = \Lop\rho,\\
\Lsgl^{-1} L \rho = -\Lsgl^{-1}\uinc.
\end{cases}
$$

\subsubsection{Dense storage}

In \mudiff, the quantity $-\Lsgl^{-1}\uinc|_{\Gamma}$ is provided by \PlaneWavePrecond whereas $\Lsgl^{-1} L$ is obtained with \PrecondDirichlet. The syntax for the dense version is then the following
\begin{lstlisting}
[...]
%Right-hand side
UincPrecond = PlaneWavePrecond(O, a, M_modes, k, beta_inc);
%Matrix of the system (the two following lines are the same)
APrecond = PrecondDirichlet(O, a, M_modes, k);
%Solving (here, directly)
rho = APrecond \ UincPrecond;
[...]
\end{lstlisting}
\medskip

\subsubsection{Sparse storage}

The sparse storage is here almost the same as for the dense version thanks to \SpPrecondDirichlet
\begin{lstlisting}
SpPrecond = SpPrecondDirichlet(O, a, M_modes, k);
%% Solving and combining operators:
rho = gmres(@(X)SpMatVec(X,M_modes, SpPrecond), UincPrecond);
\end{lstlisting}
\medskip

\section{The Neumann boundary-value problem}

Let us now consider the sound-hard scattering problem
$$
\left\{\begin{array}{r c l l}
(\Delta +k^2)u & = & 0, & \text{ in }\Omegaps,\\
\dn u & = & -\dn \uinc, & \text{ on }\Gamma,\\
\multicolumn{4}{l}{\qquad \qquad u \text{ outgoing}.}
\end{array}\right.
$$

An efficient solution to this problem is given for example by a preconditioned integral equation for sound-hard obstacles.
Here, we only present this solution but the extension to other kinds of integral equations is direct.
 The $\mu$-diff script is close to the one developed for the Dirichlet problem, only the two following functions must
  be modified: \PrecondDirichlet is replaced by \PrecondNeumann and the right-hand side \PlaneWavePrecond is now given by
   \DnPlaneWavePrecond. For the Neumann problem, the preconditioned boundary integral equation is based on the double-layer representation
$$
\begin{cases}
u = \Mop\lambda,\\
\widehat{D}^{-1} D \lambda = -\widehat{D}^{-1}\dn\uinc.
\end{cases}
$$
\begin{lstlisting}
% Three unit disks 
O = [-5, 0, 5; -2, 0, 2];
a = [1, 1, 1];
%Set the parameters...
k = 1; %wavenumber
beta_inc = 0; %incident angle
%Fourier series truncation parameter
M_modes = FourierTruncation(a, k, 'Min', 1);
%Right-hand side
DnUincPrecond = DnPlaneWavePrecond(O, a, M_modes, k, beta_inc);
%Matrix of the system (the two following lines are the same)
APrecond = PrecondNeumann(O, a, M_modes, k);
%Solving (here, direct)
lambda = APrecond \ DnUincPrecond;
\end{lstlisting}
\medskip

The post-processing is based on the double-layer potential (compared to Dirichlet, the modification is realized in the \RCS function, the
 last argument \code{[1,0]} is then replaced by  \code{[0,1]})
\begin{lstlisting}
%Scattering angles
theta_RCS = 0:360;
theta_RCS_rad = theta_RCS*2*pi/360;
%Radar Cross Section associated with the double-layer potential
myRCS = RCS(O, a, M_modes, k, theta_RCS_rad, lambda, [0,1]);
plot(theta_RCS, myRCS, 'k');
\end{lstlisting}
\medskip

\section{Mixing Dirichlet and Neumann boundary conditions}

Let us consider the following situation where we mix Dirichlet and Neumann boundary conditions. The scatterer is composed of
 $M_D$ sound-soft and  $M_N = M-M_D$ sound-hard obstacles, leading to the scattering problem
$$
\left\{\begin{array}{r c l l}
(\Delta +k^2)u & = & 0, & \text{ in }\Omegaps,\\
u & = & -\uinc, & \text{ on }\Gamma_p,\quad p=1,\ldots, M_D,\\
\dn u & = & -\dn \uinc, & \text{ on }\Gamma_p,\quad p=M_D+1,\ldots, M,\\
\multicolumn{4}{l}{\qquad \qquad u \text{ outgoing}.}
\end{array}\right.
$$
For this problem, the  preconditioned integral is not directly available. We can apply a Combined Field Integral Equation for the mixed problem
 (see equation (\ref{eq:CFIEMixte2})) and written as $$(\frac{I}{2} +A)\varphi = b,$$ where
  the matrix $A$ is
  given by  (\ref{eq:CFIEMixteApq}). We have
$$
A(p,q) = 
\begin{cases}
(1-\alpha) N^{p,q} + \alpha\eta L^{p,q}, & \text{ if } q \leq M_D,\\
(1-\alpha) D^{p,q} + \alpha\eta M^{p,q}, & \text{ if } q > M_D.\\
\end{cases}
$$
The  matrix is particularly easy to build  with \mudiff thanks to the frontal function \IntegralOperator. To this end, two
 three-dimensional arrays, \code{Assembling} and \code{Weight}, are built such that
$$
\code{Assembling}(:, p,q) = 
\begin{cases}
[4,2], & \text{ if } q \leq M_D,\\
[5,3], & \text{ if } q > M_D,
\end{cases}\qquad\text{and}\qquad
\code{Weight}(:, p,q) = [(1-\alpha), \alpha\eta].
$$
The indices in \code{Assembling} corresponds to the indices of the boundary integral operators ($2=L^{p,q}$, $3=M^{p,q}$, $4=N^{p,q}$, $5=D^{p,q}$). The assembling process is realized by \IntegralOperator.
\begin{lstlisting}
% Two Dirichlet obstacles (unit diks)
OD = [-5, -5; -5, 5];
aD = [1, 1];
N_scatD = length(aD);
% Two Neumann obstacles (unit diks)
ON = [5, 5; -5, 5];
aN = [1, 1];
N_scatN = length(aN);
%All obstacles
O = [OD, ON];
a = [aD, aN];
N_scat = N_scatD + N_scatN;
%Set the parameters...
k = 1; %wavenumber
beta_inc = 0; %incident angle
%Fourier series truncation parameter (Dirichlet, Neumann, All)
M_modesD = FourierTruncation(aD, k, 'Min', 1);
M_modesN = FourierTruncation(aN, k, 'Min', 1);
M_modes = [M_modesD, M_modesN];
%Right-hand side
Uinc = DnPlaneWave(O, a, M_modes, k, beta_inc);
DnUinc = DnPlaneWave(O, a, M_modes, k, beta_inc);
B = alpha*eta*Uinc + (1-alpha)*DnUinc;
%% Assembling
Assembling = zeros(2, N_scat, N_scat);
Weight = zeros(2, N_scat, N_scat);
for p=1:N_scatD
	for q=1:N_scat
		Assembling(:,p,q) = [4;2];
		Weight(:,p,q) = [1-alpha; alpha*eta];
	end
end
for p=N_scatD+1:N_scat
	for q=1:N_scat
		Assembling(:,p,q) = [5;3];
		Weight(:,p,q) = [1-alpha; alpha*eta];
	end
end
%Frontal function
A = IntegralOperator(O, a, M_modes, k, Assembling, Weight);
%Solving (here, direct)
density = A \ B;
\end{lstlisting}
\medskip

The post-processing is then done by specifying to \mudiff how the density must be used: for the first $M_D$ obstacles, a single-layer potential is used,
while for the others, a double-layer potential is required. The \RCS function can simply do that. It just needs an array \code{TypeOfOp} of size
 $\code{N\_scat}\times 2$ such that
$$
\code{TypeOfOp}(p,:) = \begin{cases}
[1,0], &\text{ if } p \leq M_D,\\
[0,1], &\text{ if } p > M_D.
\end{cases}
$$
In a \mudiff script, this  means ''Apply the single-layer potential (multiplied by $1$) for the first $M_D$ part of the density and a double-layer potential
 (multiplied by $1$) for the others''.
 \newpage
\begin{lstlisting}
%Preparing TypeOfOp
TypeOfOp = zeros(N_scat, 2);
for p =1:N_scat
	if(p <= N_scatD)
		TypeOfOp(p,1) = 1;
	else
		TypeOfOp(p,2) = 1;
	end
end
%Scattering angles
theta_RCS = 0:360;
theta_RCS_rad = theta_RCS*2*pi/360;
%Radar Cross Section computation
myRCS = RCS(O, a, M_modes, k, theta_RCS_rad, density, TypeOfOp);
plot(theta_RCS, myRCS, 'k');
\end{lstlisting}





\chapter{Conclusion}
\mtcaddchapter[Conclusion]                          % solution pour minitoc
\markboth{\uppercase{Conclusion}}{\uppercase{Conclusion}} 

\ldots


\medskip
\textbf{Acknowledgments.}
The second, third and fourth authors would like to thank the Institute of Scientific Research and Revival of Islamic Heritage at Umm Al-Qura University (Project ID 43405027) for the financial support. 


\bibliographystyle{plain}
\bibliography{Biblio_mudiff.bib}



\end{document}

